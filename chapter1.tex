
\chapter{Introduction}
\label{chapter1}
In this chapter I will give an introduction to my Thesis, the Navier Stokes equations and the well-popularised Projection method that solves the equations numerically.

\section{Motivation}
Fluid mechanics is the study of the motion of fluid substances, including liquid, gases and plasmas. The motion of fluid substances like in many other physics fields, can be described by a set of mathematical equations with appropriate assumptions imposed. The theory of fluid mechanics rely on the ``Continuum hypothesis" which basically treat objects in study as continuous substances. Navier Stokes equations named after Claude-Louis Navier (1785-1836) and George Gabriel Stokes (1819-1903) is arguably the most fundamental set of equations in fluid mechanics. It is a set of partial differential equations describing the general motion of fluid substances.\\

\subsection{Brief history of Navier Stokes Equations}
The theoretical development of fluid mechanics date back to the 18th centuries when important works have been published by some of the most influential scientists and mathematicians at the time. Including Leonhard Euler, Jean le Rond d'Alembert, Joseph Louis Lagrange and many more. In 1759 by applying Newton's second law of motion (which basically states the product of mass and acceleration is equal to the sum of body and external forces), Euler was able to provide a set of governing equations that describes fluid motions. Later this was known to be Euler's equations or Euler's equations of inviscid flow. It describes the motion of fluid neglecting the effect of friction forces arises between surfaces of different fluids or other material in contact. Although mathematically elegant, it has a serious drawback because as was proved by D' Alembert and Euler himself that the drag force acting on a body moving in a steady fluid is always zero. This is in contradiction with real life experimental results. Later this was proven to be the lack of viscosity term in the equations. Hence that's why the theory is only for ideal flows where the viscosity is zero. \\

In the 19 century, several scientists have worked to fill the gap between theory and experimental observations. In 1822 Claude Louis Marie Henri Navier provided a set of more generalised equations on fluid motions. This was considered to be a major break through, however it was derived by the theory of molecular attraction and repulsion. The equations were re-derived many times by Cauchy (1828), Pisson (1829) and Saint-Venant (1843) each from a rather different point view. Later in 1845 George Gabriel Stokes re-derived again by the introduction of viscosity. More history about the origin of Navier Stokes equations can be found in a beautiful paper by Olivier Darrigol \cite{darrigol2002between}.\\

Although derived from concrete physics principles, Navier Stokes equations at a mathematical level is still not complete. The existence and uniqueness of smooth solutions in 3-D is yet to be proven (or disproven). This is really a bottleneck for the theory. At the turn of the 20th century, especially through the development of computer digital power, the numerical solutions of Navier Stokes equations seem to provide a viable alternative. In fact, one of the founding fathers of Computational Fluid Dynamics (CFD), John Von Neumann has predicated that the computer simulated solutions would eventually replace analytical solutions and make the laboratory experiments obsolete. Indeed his predication did not come true completely, nevertheless the study of computer generated numerical solutions do become increasingly important.\\

The computation of numerical solutions for Navier Stokes equations are actually not very straightforward to do so because the pressure and velocities are coupled with each other. Solving velocities and pressure individually requires the knowledge of each other. Thus they have to be solved together. Not only this brings heavy computational effort, but also the linear system resulted leads to a ``saddle" point problem. Especially in the first half of last century when the computing power was much less powerful as what it stands today. Alternative ways to compute the numerical solutions were strongly needed at the time.\\



\subsection{Origin of Projection method}
Alexandre Joel Chorin and Roger Temam in late 1960s have independently proposed a very clever alternative method which now known as ``Projection method". It basically decouples the pressure and velocities so that each variable can be solved independently, thus avoid the difficulties the coupled equations have. This was proven to be a very robust and efficient method. To date, this is still the popular method which most modern Navier Stokes solvers rely on. However the improvement in efficiency comes at a price. The accuracy of Projection methods is often questioned, especially for the pressure variable. The accuracy of velocity variable can be easily improved whereas pressure is not. In fact many researchers have argued that the numerical pressure can only be first order accurate in time. Although many recent papers have shown second accurate schemes, however this is still an open question. Some papers even show slight contradictory results in the literature \cite{brown2001accurate, pyo2005normal,guermond2004error,guermond2006overview}. In this thesis, we examines the second order accurate extensions of Chorin's original projection methods and provide theoretical and numerical investigations on their overall accuracy.

\section{Outline of Thesis}
Here I will give a basic outline of my thesis. The first chapter is an introduction to Navier Stokes equations and Projection method which is widely used in computing numerical solutions. The second chapter presents the derivation of Navier Stokes equations from first physics principles. In chapter 3 we first talk about the limitation of the coupled solver and then give a detailed explanation of Projection methods, including its original idea from the theory of Helmholtz Hodge decomposition; application of the theorem to the formation of Chorin's original projection method and recent second order modifications. In chapter 4 we give a brief analysis on the accuracy of projection methods through normal mode analysis whereas chapter 5 talks about the numerical implementation of our Navier Stokes solver; finally we present our results about the performance of projection methods through numerical validations in chapter 6. Conclusions about accuracy of projection methods are then made in chapter 7.
