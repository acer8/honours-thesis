\chapter{Properties of Navier Stokes Equations}
\label{chapter3}

\section{Navier Stokes Equations as initial value problem}
Talk about the weak form of NSE

\begin{itemize}
\item NSE is scale invariant, Picard contraction principle
\end{itemize}

\section{Stokes equations}
Stokes equations are essentially the linearised Navier Stokes Equations. It is when the advective inertial force is very small compared to viscous force ($\textbf{Citation needed!}$). Hence it describes a very viscous flow with Reynolds number << 1.\\

Due to its simplicity it is often used in proving uniqueness and existence of solutions\\
($\textbf{citation needed, this part not sure}$ as well as error analysis of numerical solvers \cite{shen1992error,brown2001accurate}.\\

The Stokes equations:
\begin{dgroup}
\begin{dmath}
\mu \nabla^2 \textbf{u} - \nabla p = f
\end{dmath}
\intertext{where $\mu$ is the dynamic viscosity and $f$ is a forcing representing an external force, such as Gravity.\\
With the usual divergence constraint}
\begin{dmath}
\nabla \cdot \textbf{u} = 0
\end{dmath}
\end{dgroup}

If we replace the operators with its numerical discretisations then we can write Stokes Equations in block matrix form.\\

\begin{center}
Let's $A$ denote the Stokes operator:
\begin{equation}
A = \mathbb{P} (\mu \nabla^2 \textbf{u})
\end{equation}
\end{center}
where $\mathbb{P}$ is the projection operator which projects vector fields into the divergence free vector fields. We will study this in detail in Chapter 4.

\begin{center}
Let $D$ denote the divergence operator which basically approximates the divergence $\nabla \cdot$.\\
By construction $D$ is skew adjoint such that 
\begin{equation}
D = -G^T
\end{equation}
where $G$ is the gradient operator.
\end{center}

Hence we can rewrite the Stokes equations as:
\begin{dgroup}
\begin{dmath}
Au + D^T p = f
\end{dmath}
\begin{dmath}
Du = 0
\end{dmath}
\end{dgroup}
and in block matrix form:
\begin{equation}
\begin{bmatrix}
A & D^T \\
D & 0\\
\end{bmatrix}
= \begin{bmatrix}
\textbf{u}\\
\textit{p}
\end{bmatrix}
= \begin{bmatrix}
\textit{F}\\
0
\end{bmatrix}
\end{equation}


\section{Role of Pressure}
Pressure is a rather mysterious variable.\\
We can approach the Navier Stokes Equations from an optimisation point of view.\\

First we will cover some background of optimisation.\\

Suppose we have an objective function $J (\textbf{x}$ with $\textbf{x} \in L^2 (\Omega)$ (e.g. $\mathbb{R}$) subject to a constraint $g(\textbf{x}) = c$.\\
We want to optimise (to maximise or minimise) $J (\textbf{x}$ given $g(\textbf{x}) = c$.\\

We could use the Lagrange optimisation approach. \\
The extrema of $J$ occurs at a point where it should not increase or decrease in any direction near the contour line $g (\textbf{x} = c$. Hence this is only possible if the gradient of $J$ is parallel to the gradient of $g$, or we have $\nabla J = 0$. Based on this observation we can define an auxiliary Lagrangian function as:


\begin{equation}
\Lambda (\textbf{x}, \lambda) = J (\textbf{x}) + \lambda (g (\textbf{x}) - c)
\end{equation}
where $\lambda \in \mathbb{R}$ is a constant called Lagrangian multiplier.\\
Hence we can find the extrema of $J$ by finding the critical points of the Lagrangian function
\begin{dmath}
\nabla \Lambda = 0
= \nabla J (\textbf{x}) + \lambda \nabla g (\textbf{x})
\end{dmath}
By solving a simultaneous equations for $\lambda$ and $\textbf{x}$ we can recover the extrema of $J$ subject to the constraint $g$.\\

For Navier Stokes Equations and especially the linearised Stokes equations, we can treat this as an optimisation problem.\\

We want to minimise the kinetic energy of the fluid (the objective function) subject to the divergence constraint.
\begin{dgroup}
\intertext{minimise}
\begin{dmath}
J = \dfrac{1}{2} ||A \textbf{u}||^2 - F^T \textbf{u}
\end{dmath}
\intertext{\textbf{need to double check this!}\\
subject to the divergence constraint}
\begin{dmath}
D \textbf{u} = 0
\end{dmath}
\end{dgroup}
Hence the Lagrangian function is:
\begin{equation}
\Lambda (\textbf{u}, \lambda) = \dfrac{1}{2} ||A \textbf{u}||^2 - F^T \textbf{u} + \lambda D \textbf{u} 
\end{equation}

By taking $\nabla \Lambda = 0$ we obtain:
\begin{dmath}
A \textbf{u} - F + D^T \lambda = 0
\end{dmath}
and combine with the divergence constraint we obtain (in block matrix form):

\begin{equation}
\begin{bmatrix}
A & D^T \\
D & 0\\
\end{bmatrix}
= \begin{bmatrix}
\textbf{u}\\
\lambda
\end{bmatrix}
= \begin{bmatrix}
\textit{F}\\
0
\end{bmatrix}
\end{equation}
Hence it is obvious that the pressure variable is actually the Lagrange multiplier of the optimisation problem.
$\textbf{need further reading to extend this idea!}$


