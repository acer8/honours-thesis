
\chapter{Error Analysis of Projection methods}
\label{chapter5}
In this chapter we will present the error analysis for different ``second order" Projection methods and the Gauge method through rigorous normal mode analysis.

\section{Introduction of Normal mode analysis}

There are many ways to analyse the convergence and error for numerical methods, including the popular Energy methods \cite{liu1996projection,guermond2006overview} and Normal mode methods. Although the process of these methods differ but their essence is the same: compare the numerical solutions obtained to the true analytical solutions\cite{pyo2005normal,guermond2004error}. Energy methods can be used in a general setting but depends on the particular error structure of the projection methods and sometimes difficult to obtain \cite{guermond2006overview}. Other other hand, normal mode analysis is simple and reveals more precise information of the error of the projection methods. Hence recently there has been an increase of using normal mode analysis in the literature. However there is a major limitation because the analysis can only be performed on special domains and boundary conditions \cite{strikwerda1999accuracy,pyo2005normal,brown2001accurate}. For instance, in Brown. et. al 2001 paper \cite{brown2001accurate}, only one normal mode was considered and hence the results could not be easily generalised.\\
Nevertheless, due to its simplicity and precision we will still perform a simple normal mode analysis to the ``second order" projection methods in this section, but first let's go through some backgrounds.\\

A normal mode describes a particular motion of an oscillating system where all the components of the system are moving in the same frequency and also with the same phase. The centre of the mass often remains stationary. The frequency of these normal modes is often coined as the resonant or natural frequency of the system. The motion of any physical objects can be characterised by their set of normal modes. For instance, a guitar, a spring or even buildings can vibrate in their natural frequencies. In fact, by the superposition principle, the general motion of a system is a superposition (linear combination) of its normal modes. Normal modes is a natural characteristic of the physical object which depends on the material, structure and boundary conditions of the system. It is called "Normal" because the normal modes are orthogonal to each other which means that their inner product is zero. As a result, the normal mode motions are often independent of each other in the sense that the excitation of one does not affect the motion of others. One of the simplest example is the vibration of two masses joined by springs. The system is characterised by a set of 2 differential equations relating the acceleration of the masses with the forces acting on them. The normal modes are simply the solution to these differential equations. In this example, the two masses could move in the same direction corresponds to symmetric motions or moving opposite to each other which corresponds to an antisymmetric motion. Of course they are still moving in the same frequencies.\\

\begin{figure}[H]
	\centering
	\includegraphics[width=4.5in]{normal_mode_2_mass_system.pdf}
	\caption{Picture of 2 moving mass system}\label{fig:5.1}
\end{figure}
Because the fluid is of course a physical object and hence will have a set of normal modes which can be obtained by solving the Navier Stokes equations. However because the Navier Stokes equations cannot be solved analytically in most cases. Hence we will need to consider special domains. For instance, a semifinite periodic channel or a half plane are the two most simplest examples \cite{strikwerda1999accuracy}. We will then Laplace and Fourier transform the equations to reduce to a system of Ordinary Differential equations which can be solved easily. By repeating the same procedure for our numerical methods we can compare the numerical solutions with the reference solution and thus obtaining an error bound for the primitive variables.

\section{Normal mode analysis of Projection methods}
Here is the actual analysis.

We are mainly concerned with the error in time stepping. The interaction of boundary conditions, pressure update and the second order implicit viscous term are more important to us. Thus the advective term can be neglected. \cite{brown2001accurate,strikwerda1999accuracy,pyo2005normal,guermond2004error,liu1996projection,shen1996error,shen1992error}. This has been done in most of the textbooks and papers, and hence we will follow the convention and perform the analysis on the linearised unsteady Stokes equations.\\

We perform the analysis in a similar fashion to what has been done in David Brown paper \cite{brown2001accurate} and work with a periodic semi-infinite channel which is one of the simplest settings to consider slip boundary conditions.

\textbf{Periodic semi-infinite channel}
Errors in interior points can often easily be improved whereas it is more difficult to do so along the boundary. A physical boundary (e.g. no-slip boundary condition) could cause the error to behave differently. In most of the numerical studies (including this project) the flow is often confined in a bounded region, where slip-conditions are commonly used. For instance, a square domain with 4 sides. It is interesting to see how the Projection methods performs under these slip-boundary conditions. For the sake of simplicity we consider rather a microscopic picture of such domains where we focus on one side of domain only. A Dirichlet boundary condition is imposed on it whereas other parts are left free. This is like a 1-dimensional flow in a 2-dimensional plane. We achieve this by using a semi-infinite periodic channel where the x direction flow is non-trivial but the y direction flow is fixed to be periodic in time. This is essentially the same as the one used in Brown's paper \cite{brown2001accurate}. While this geometry gives us an initial idea of the error behaviour for projection methods but it is too special and non-physical because we ignore the interaction between different boundaries. Whether error would behaves the same for other kinds of domains remains unanswered \cite{pyo2005normal}. \\

The domain is explicitly stated as:
\begin{equation*}
\Omega = \left[0, \infty \right) \times \left[0, 2\pi\right], \, \, \, t \in [0, T]
\end{equation*}
where $T$ is an arbitrary end time.\\
and boundary points:
\begin{equation*}
\partial \Omega = \{(x,y) \in \mathbb{R}^2: x = 0\} \cup \{(x,y) \in \mathbb{R}^2: y = 0, \, 2\pi\}
\end{equation*}

A Dirichlet boundary condition is imposed on the left end of the channel ($x = 0$) and the flow in y is assumed to be equal to the top and bottom boundaries to ensure periodicity ($\textbf{u}(x,0,t) = \textbf{u}(x,2\pi,t)$). (\textbf{a picture of the channel is needed!})\\

Let's first compute the reference solution.\\
The 2 dimensional ($u(x, y, t), v(x, y, t), p(x, y, t)$) initial value problem of the linearised Stokes equations is presented below. We assume $u(x, y, t), v(x, y, t), p(x, y, t)$ are solutions to the Stokes equations and they satisfy both the momentum and continuity equations.
\begin{dgroup}
\begin{dmath}
\textbf{u}_t + \nabla p = \dfrac{1}{R} \nabla^2 \textbf{u} \condition{in $\Omega$}
\end{dmath}
\begin{dmath}
\nabla \cdot \textbf{u} = 0 \condition{in $\Omega$}
\end{dmath}
\intertext{where $R$ again denotes the Reynolds number\\
boundary conditions:}
\begin{dmath}
u(0,y,t) = \alpha (y,t)\condition{in $\partial \Omega$}
\end{dmath}
\begin{dmath}
v(0,y,t) = \beta (y,t) \condition{in $\partial \Omega$}
\end{dmath}
\intertext{initial value}
\begin{dmath}
\textbf{u} (x,y,0) = \textbf{u}_0
\end{dmath}
\intertext{For simplicity, let's assume $\textbf{u}_0 = 0$\\
by taking the divergence to the momentum equation given above, we arrive at another important condition}
\begin{dmath}
\nabla^2 p = 0 \condition{in $\Omega$}
\end{dmath}
\end{dgroup}

Because of the periodic geometry, we are only concerned with the fluid motion in x direction. Thus we regard the flow as a ``1-D" flow in x. We can then reduce the problem into ordinary differential equations (ODE) by taking Laplace transform in time and Fourier transform in the y direction. Taking the transform in x direction will yield the same result (of course this corresponds to periodic flow in x instead). Let $s$ and $k$ denote the Laplace and Fourier transform variables respectively.

\begin{dgroup*}
\intertext{The unilateral Laplace transform is given as\\
}
\begin{dmath*}
\mathcal{L} [f(t)] = \int_0^\infty f(t) e^{-st} dt
\end{dmath*}
\intertext{$f(t)$ is an integrable function defined over the interval $[0, \infty]$\\
\textbf{Check this, maybe don't need this sentence.}}
\intertext{and the standard Fourier transform (with angular frequency $k$) is defined as\\
}
\begin{dmath*}
\mathcal{F} [f(y)] = \dfrac{1}{\sqrt{2 \pi}}\int_{-\infty}^{\infty} f(y) e^{-iky} dy
\end{dmath*}
\end{dgroup*}

We take Laplace transform followed by Fourier transform to the linearised Stokes equations. Let a hat ($\hat{.}$) denote the final transformed variable.\\
By the common Fourier and Laplace transform identities, we obtain the following transformed quantities.\\
\begin{equation*} 
\textbf{u} \rightarrow \hat{\textbf{u}} = (\hat{u}, \hat{v}), \, \, \, p \rightarrow \hat{p}
\end{equation*}
\begin{equation*}
\textbf{u}_t \rightarrow s \hat{\textbf{u}}, \, \, \, \nabla \cdot \textbf{u} \rightarrow \partial_x \hat{u} + ik \hat{v}
\end{equation*}
\begin{equation*}
\nabla^2 \textbf{u} \rightarrow (\partial_x^2 -k^2)\hat{\textbf{u}}, \, \, \,  \nabla p \rightarrow (\partial_x p, ik p)^T
\end{equation*}

Hence the transformed Stokes equations are:
\begin{dgroup}
\begin{dmath}
(\partial_x^2 - \mu^2) \hat{u} = R \, \partial_x \hat{p} \condition{in $\Omega$}
\end{dmath}
\begin{dmath}
(\partial_x^2 - \mu^2) \hat{v} = ik \, R \, \hat{p} \condition{in $\Omega$}
\end{dmath}
\begin{dmath}
(\partial_x^2 -k^2) \hat{p} = 0 \condition{in $\Omega$}
\end{dmath}
\begin{dmath}
\partial_x \hat{u} + ik \hat{v} = 0
\end{dmath}
\begin{dmath}
\hat{u} (0,y,t) = \hat{\alpha} \condition{in $\partial \Omega$}
\end{dmath}
\begin{dmath}
\hat{v} (0,y,t) = \hat{\beta} \condition{in $\partial \Omega$}
\end{dmath}
\end{dgroup}
%Thus we obtain: $\partial t -> s, \, \, \, \partial y -> ik$
where $\mu^2 \equiv k^2 + R \, s$ and we take $\mu$ to be the positive real part of the solution for simplicity \cite{brown2001accurate}.\\

Because Equations (5.2) are ordinary differential equations, hence we can solve them easily using method of characteristics. Let's start from the pressure equation (E (4.2 c)).\\
Its characteristic equation is given as:

\begin{equation*}
r^2 - k^2 = 0
\end{equation*}
Solving it we obtain:
\begin{equation*}
r = \pm |k|
\end{equation*}
By superposition principle we obtain:
\begin{equation*}
\hat{p} = P_1 e^{|k| x} + P_2 e^{- |k| x}
\end{equation*}

Hence there are 2 distinct modes of motion: exponential growth and decay. However because of our infinite half plane geometry, we would see the pressure (and also velocities) grow to infinity as $x \rightarrow \infty$. This is physically impossible and we don't want to see our solutions blow up in space! Therefore the exponential growth mode must be dropped, leaving the solution as:

\begin{equation}
\hat{p} = P e^{-|k| x}
\end{equation}
where $P$ is a constant amplitude to be determined.\\
Hence we are left with only one normal mode. In fact, this exponential decaying motion is also referred as ``quasi-normal mode" in literature \textbf{Wikipedia, I think this concept is right here?}. \\

Thus we have seen the limitation of normal mode analysis already because the normal modes strongly depends on the spatial domain of the problem.\\

Substitute the expression for $\hat{p}$ into E (5.2 (a)) we obtain an equation for $\hat{u}$:
\begin{dgroup}
\begin{dmath}
\partial_x^2 \, \hat{u} - \mu^2 \, \hat{u} = -|k| \, Re \, P e^{-|k| x}
\end{dmath}
\intertext{\\
this is again an (inhomogeneous) ordinary differential equation for $\hat{u}$! \\
Solving it we obtain:\\
}
\begin{dmath}
\hat{u} = U e^{-\mu x} + \dfrac{|k|}{s} P e^{-|k| x}
\end{dmath}
\end{dgroup}
where once again the exponentially growing mode ($e^{-\mu x}$) is dropped out.\\

Similarly by substituting the expression of pressure (Equation 5.3) into the ODE for $\hat{v}$ (Equation 5.2 b) we obtain the solution for $\hat{v}$:
\begin{equation}
\hat{v} = V e^{-\mu x} - \dfrac{ik}{s} P e^{-|k| x}
\end{equation}

And by the divergence constraint E (5.2 (d)) we obtain another equation relating $\hat{u}$ and $\hat{v}$
\begin{dmath}
- \mu U + ik \, V = 0
\end{dmath}

Now with 3 equations corresponding to 3 unknowns, we can solve for $U, V$ and $P$ by applying the boundary conditions given in Equation 5.2 (d-f). The problem is organised neatly in the following matrix form:

\begin{equation}
\begin{bmatrix}
1 & 0 & |k| / s \\
0 & 1& - i k / s \\
-\mu & ik & 0 \\
\end{bmatrix}
\begin{bmatrix}
U\\
V\\
P\\
\end{bmatrix}
= \begin{bmatrix}
\hat{\alpha}\\
\hat{\beta}\\
0\\
\end{bmatrix}
\end{equation}

Solving the matrix problem and we obtain the coefficients for the solutions as:

\begin{dgroup}
\begin{dmath}
U = \dfrac{(\mu + |k|)}{R \, s} (- |k| \hat{\alpha} + ik \hat{\beta})
\end{dmath}
\begin{dmath}
V = \dfrac{-i (\mu + |k|)\mu}{R \, s} (-\dfrac{k}{|k|} \hat{\alpha} + i \hat{\beta})
\end{dmath}
\begin{dmath}
P = \dfrac{(\mu + |k|)}{R \, |k|} (\mu \hat{\alpha} - ik \hat{\beta})
\end{dmath}
\end{dgroup}

And finally we arrive at the reference solutions for our initial value problem of the linearised Stokes equations: 

\begin{dgroup}
\begin{dmath}
\hat{u}_{ex} = \dfrac{(\mu + |k|)}{R \, s} (- |k| \hat{\alpha} + ik \hat{\beta})e^{-\mu x} + \dfrac{(\mu + |k|)}{R \, s} (\mu \hat{\alpha} - ik \hat{\beta}) e^{-|k| x}
\end{dmath}
\begin{dmath}
\hat{v}_{ex} = \dfrac{-i (\mu + |k|)\mu}{R \, s} (-\dfrac{k}{|k|} \hat{\alpha} + i \hat{\beta}) e^{-\mu x} - \dfrac{i k (\mu + |k|)}{R \, s \, |k|} (\mu \hat{\alpha} - ik \hat{\beta}) e^{-|k| x}
\end{dmath}
\begin{dmath}
\hat{p}_{ex} = \dfrac{(\mu + |k|)}{R \, |k|} (\mu \hat{\alpha} - ik \hat{\beta}) e^{-|k| x}
\end{dmath}
\end{dgroup}
where the subscript ``ex" represents exact solutions.\\

\subsection{Normal analysis for Algorithm 1,2,3}
Now let's perform the above analysis to the numerical schemes described in the previous chapter. \\
Note that the numerical schemes we work with are only discretised in time because we are mainly concerned in the error behaviour in time stepping rather than spatial stepping. Therefore we will need to use discrete Laplace transform whereas keep using the continuous Fourier transform.\\

with a time step of $\Delta t$ the discrete Laplace transform (or Z transform) is defined as \cite{strikwerda1999accuracy}:
\begin{equation*}
X(z) = \Delta t \sum_{n=-\infty}^{n=\infty} z^{-n} \, x(n), \text{     $z = e^{s \Delta t} $ and $|z| > 1$}
\end{equation*}
\cite{strikwerda1999accuracy, brown2001accurate, pyo2005normal}\\
Note $z$ is the discrete transform variable (not the z - direction) and it is related to the continuous transform by: $z = e^{s \Delta t}$.\\

The Z transform converts a series of discrete time signals $x(n)$ into complex frequencies domain. Hence $X^{n+1}$ is merely $X^n$ shifted up by $z$. To see this define a new set of signals $w$ being basically the original signal $x(n)$ at n+1 step. Hence $w^n = x^{n-1}$. Then by applying 
\begin{dmath}
X^{n+1} = W^n
= \Delta t \sum_{n=-\infty}^{n=\infty}\,w^n\,z^{-n} =  \Delta t \sum_{n=-\infty}^{n=\infty}\,x^{n-1}\,z^{-n} 
=  \Delta t \sum_{n=-\infty}^{n=\infty}\,x^n\,z^{-(n+1)} = z \Delta t \sum_{n=-\infty}^{n=\infty}\,x^n\,z^{-n} 
= z X^n
\end{dmath}

As discussed before there are 3 important things that are crucial to the accuracy of projection methods: $q$: choice of pressure approximation; boundary condition of the intermediate velocity field and choice of the non-physical variable $\phi$ resulted from projection. David has argued that the coupling between these 3 things must be considered to achieve second order accuracy for all primitive variables. We will also show this in the subsequent analysis \cite{brown2001accurate}\\

Let's consider a general second order projection method with centred finite difference used to approximate the time derivative and 2nd order Crank-Nicholson method used to treat the implicit diffusive terms. This is just the linearised scheme considered in chapter 4.4.2 (Equation 4.23 to 4.25)\\

The algorithm is summarised below\\
\begin{equation}
\begin{cases}
\dfrac{(\textbf{u}^* - \textbf{u}^n)}{\Delta t} + \nabla q = \dfrac{R}{2} \nabla^2 (\textbf{u}^* + \textbf{u})\,\,\,\text{   solve for $\textbf{u}^*$}\\
\textbf{u}^* \,|_{\partial \Omega}= \left(\textbf{u}^{n+1} + \Delta t \phi^{n+1}\right)|_{\partial \Omega}\,\,\,\text{   Boundary condition and projection}\\
p^{n+1/2} = q + L \phi^{n+1}\,\,\,\text{   Pressure update}
\end{cases}
\end{equation}
Where $q, \,L$, boundary conditions for projection and $\phi^{n+1}$ depends on the particular projection scheme.\\

Let's first consider the relation between $q$ and $\phi^{n+1}$\\
In Projection methods, the intermediate velocity field and auxiliary field are approximations to the true velocity and pressure, although the degree of accuracy is dependent on different methods. Hence we expect a close relationship between $\phi$ and pressure. In fact, in order to eliminate the potential numerical boundary layers that might arise in computations, David has proposed a simple definition of $\phi$:\\
Define:
\begin{equation*}
q^n = Q(n) \phi^n
\end{equation*}
Apply Laplace and Fourier transform:
\begin{equation}
\hat{q}^n = Q(z) \hat{\phi}^n
\end{equation}
Where $Q(z)$ is a function of Z transform variable that describes the coupling between the choice of pressure approximation ($q$) and the auxiliary field ($\phi$). This function indeed varies for different variations of projection methods. We will see shortly that this limit the choice of $q$ too.\\

In this section, we consider the 3 widely used Projection methods discussed in previous chapter.

\textbf{Alg 1} - Projection with Lagged Pressure term (first order update formula)

As discussed in chapter 4.4.2 this method uses a first order pressure update formula:
\begin{equation*}
p^{n+1/2} = p^{n-1/2} + \phi^{n+1}
\end{equation*}
Hence this method corresponds to
\begin{equation*}
q = p^{n-1/2}, \, \, \, L = I \text{   Identity matrix   }
\end{equation*}
And boundary condition for projection:
\begin{equation*}
\textbf{u}^* |_{\partial \Omega} = \textbf{u}^{n+1} |_{\partial \Omega}
\end{equation*}

\textbf{Alg 2} - Projection with Lagged Pressure term (second order update formula)

This corresponds to a second order pressure update formula used to improve the accuracy for Alg 1

\begin{equation*}
p^{n+1/2} = p^{n-1/2} + \nabla \phi^{n+1} - \dfrac{\Delta t}{2 Re} \nabla^2 \phi^{n+1}
\end{equation*}
\begin{equation*}
q = p^{n-1/2}, \, \, \, L = I - \dfrac{\Delta t}{2 Re} \nabla^2
\end{equation*}
And same boundary condition for projection\\

\textbf{Alg 3} - Pressure free projection
First proposed by Kim and Moin. Pressure is can be recovered using the same second order update formula as in A.g 2. \cite{kim1985application,brown2001accurate}
\begin{equation*}
q = 0, \, \, \, L = I - \dfrac{\Delta t}{2 Re} \nabla^2
\end{equation*}
Boundary condition for projection:
\begin{equation*}
\textbf{n} \cdot \textbf{u}^* |_{\partial \Omega} = \textbf{n} \cdot \textbf{u}^{n+1} |_{\partial \Omega}, \, \, \, \textbf{$\tau$} \cdot \textbf{u}^* |_{\partial \Omega} = \textbf{$\tau$} \left(\textbf{u}^{n+1}+ \Delta t \textbf{$\tau$} \nabla \phi^{n+1}\right) |_{\partial \Omega}
\end{equation*}

Now Let's perform the Normal Mode analysis.\\
Taking divergence of Equation 5.11 (a) we can eliminate the divergence free velocity field. Note that the Laplace of $\textbf{u}^n$ is also divergence free because $\nabla \cdot \nabla^2 \,\textbf{u}^n = \nabla^2 \,\nabla \cdot \textbf{u}^n = 0$. Recall the definition of $q$ in equation 5.12 we obtain an equation between $\textbf{u}^*$ and $\phi$.

\begin{equation}
\nabla \cdot \textbf{u}^* + \Delta t \,Q(n)\,\nabla^2 \phi^{n+1} = \dfrac{\Delta t}{2} \nabla^2 \,(\nabla \cdot \textbf{u}^*)
\end{equation}

Also taking divergence of the compatibility equation (Equation 5.11 (b)) and substitute into the equation above we obtain an equation in terms of $\nabla^2\phi^{n+1}$, denote this as $\eta$
\begin{equation*}
\eta + Q(n)\eta = \dfrac{\Delta t}{2}\, \nabla^2 \eta
\end{equation*}
The variable $\eta$ describes the coupling between the intermediate velocity field $\textbf{u}^*$ and the scalar potential $\phi$ resulted from the projection. Therefore even though velocity and pressure are now decoupled from projection, their coupling is now represented by the intermediate velocity and the auxiliary variable $\phi$. Hence it is important to understand the structure of $\eta$ in order to understand the error in velocity and pressure for these projection methods.\\

Similar to the reference solution derivation, we take Laplace and Fourier transform to the equation above. After rearranging we obtain:
\begin{equation}
\partial_x^2 \hat{\eta} - \hat{\eta} (k^2 + \dfrac{2}{\Delta t}\,(1+Q(z))) = 0
\end{equation}
Further define $\gamma^2 = k^2 + \dfrac{2}{\Delta t}\,F(z)$ where $F(z) = 1 + Q(z)$ for convenience.\\
By solving the ODE for $\hat{\eta}$ we found
\begin{equation}
\hat{\eta} = A\,e^{-\gamma x}
\end{equation}
Note the non-physical exponential growing mode $e^{\gamma x}$ has already been dropped.\\

The decaying rate $\gamma$ causes trouble because the $\Delta t$ in the denominator of $\gamma$ would cause the mode to become ill-defined as $\Delta t$ approaching zero. This makes $A\,e^{-\gamma x}$ a spurious mode. \cite{brown2001accurate, strikwerda1999accuracy}. \textbf{The numerical boundary layer} and its effects on accuracy is well studied by the work of \emph{E and Liu et al} \cite{liu1996projection}. \\
It is worth noting that the spurious mode obtained by solving $\eta$ comes from the compatibility equation (Equation 5.11 (b)). Hence it is an inherent property for all projection methods. Also recall the definition of $\eta$, it is customary to think the divergence of intermediate velocity ($\nabla \cdot \textbf{u}^*$) and the auxiliary field ($\phi$) would contain the spurious mode too. We will shortly demonstrate this claim. Therefore it is of interest to devise a method which ensures the actual velocity ($\textbf{u}$) and pressure ($p$) does not contain it. \\

With the expression for $\eta$ we can now solve for $\phi$ based on the second order inhomogeneous ordinary differential equation below:
\begin{equation}
(\partial_x^2 - k^2)\hat{\phi} = \eta = A\,e^{-\gamma x}
\end{equation}
\begin{equation}
\hat{\phi} = \hat{\phi}_h + \hat{\phi}_p = A_1\,e^{-|k|x} + A_2 \,e^{-\gamma x}
\end{equation}
Where the subscript $h$ and $p$ denotes homogeneous and particular solutions from the ODE respectively.\\
We observe that $\phi_h$ contains the actual piece of solution we want whereas $\phi_p$ is not because it contains the spurious mode. Thus we have demonstrated our claim that the auxiliary variable $\phi$ contains the spurious mode. However as we shall see later that we do want $\nabla \cdot \textbf{u}^*$ converges to zero as $\Delta t \rightarrow 0$ and then the homogeneous solution would dominate.\\
Hence we can now solve the intermediate velocity field by taking the divergence of compatibility condition:
\begin{equation}
\nabla \cdot \hat{\textbf{u}^*} = \Delta t \,(\partial_x^2 - k^2)\hat{\phi} = \Delta t\,A_2 e^{-\gamma x}\,(\gamma^2 - k^2)
\end{equation}
Because $\gamma^2 - k^2 = F(z) = 1 + Q(z)$ and this term obviously would not be zero for any of the projection methods, hence we have demonstrated that the divergence of the intermediate velocity field also contains the spurious mode. These findings are consistent with Brown, strikwerda and Shen \cite{brown2001accurate, strikwerda1999accuracy}.\\

Now it is interesting to see how the spurious mode would affect the accuracy for velocity and pressure in the projection methods. \\

For Alg 1, the pressure update formula is:
\begin{equation}
p^{n+1/2} = q + \phi^{n+1} = (Q(n) + 1)\phi^{n+1}
\end{equation}
Where the lagged pressure approximation $q = p^{n-1/2}$ is used. \\
\begin{equation}
\hat{p}^{n+1/2} = (Q(z)+1)\hat{\phi}
\end{equation}
Then by recalling the relation between $q$ and $\phi$ (Equation 5.12) we found that:
\begin{equation*}
\hat{q} = Q(z)\,\hat{\phi} = \hat{p}^{n-1/2} = \dfrac{1}{z^{3/2}}\,\hat{p} = \dfrac{1}{z^{3/2}} \, z^{1/2} (Q(z) + 1) \hat{\phi}
\end{equation*}
Then we obtain
\begin{equation*}
Q(z) = \dfrac{1}{z} (Q(z) + 1)
\end{equation*}
This finally leads to the following expression of $Q(z)$:
\begin{equation}
Q(z) = \dfrac{1}{z-1}
\end{equation}
Hence we found that the pressure is updated as:
\begin{equation}
\hat{p}^{n+1/2} = \dfrac{1}{z-1}\,(A_1\,e^{-|k|x} + A_2 \,e^{-\gamma x}) + A_1\,e^{-|k|x} + A_2 \,e^{-\gamma x}
\end{equation}
The clearly indicates the presence of spurious mode. Hence the the accuracy is strongly limited due to formation of numerical boundary layer. The boundary layer not only causing degradation in accuracy along the boundary but could also affect the nearby interior points too. In some domains (like a squared domain with Drichlet boundary condition imposed), the spurious mode would cause strong oscillations along the boundary.\\

In addition as the projection implies, the auxiliary field often satisfies a normal Neumann boundary condition :
\begin{equation*}
\textbf{n} \cdot \nabla \phi^{n+1} = \dfrac{\partial \phi^{n+1}}{\partial n} = 0
\end{equation*}
This is often referred as a non-physical boundary condition in the literature \cite{strikwerda1999accuracy, guermond2004error, brown2001accurate} for the reasoning discussed below.\\
Extracting the normal component of the gradient of pressure along the boundary (Equation 5.19) we found:
\begin{equation}
\textbf{n} \cdot \nabla \hat{p}^{n+1/2}\,|_{\partial \Omega} = \textbf{n} \cdot \nabla \hat{p}^{n-1/2}\,|_{\partial \Omega}
\end{equation}
This implies the normal pressure gradient is constant along the boundary for all time steps. This is in general not consistent with the actual boundary condition of pressure gradient. Hence this partly explains why numerical boundary layer exists.\\

The other update formula used for Alg 2 and 3 should show an improved accuracy.\\
The transformed version of the formula is:
\begin{equation}
\hat{p}^{n+1/2} = \dfrac{1}{z^{1/2}}\hat{p} = Q(z)\hat{\phi} + \hat{\phi} - \dfrac{1}{2}\hat{\nabla \cdot \textbf{u}^*}
\end{equation}
substitute the expression for $\hat{\nabla \cdot \textbf{u}^*}$ and $\hat{\phi}$ into the equation above we obtain:
\begin{equation}
\dfrac{1}{z^{1/2}}\hat{p} = Q(z)\,A_1\,e^{-|k|x} + Q(z)\,A_2 \,e^{-\gamma x} + A_1\,e^{-|k|x} + A_2 \,e^{-\gamma x} - (1+Q(z)) \,e^{-\gamma x}
\end{equation}
After rearranging we found that the spurious mode $e^{-\gamma x}$ is actually being cancelled out, leaving the pressure as:
\begin{equation}
\hat{p} = z^{1/2}\,(Q(z) + 1)\,A_1\,e^{-|k|x} 
\end{equation}
where the $Q(z)$ varies depending on $q$. It could be $\dfrac{1}{z-1}$ as in $Pm\,1\,(a)$ or zero in $Pm\,2$. However in either case, the spurious is filtered out from the pressure update equation. Here from this point of view, this formula shows an improved accuracy.\\

However this analysis needs more attention for $Pm\,1\,(b)$, because the pressure approximation uses a lagged pressure value and recall the relation between $q$ and $\phi$ we find:
\begin{equation*}
p^{n-1/2} = Q(n)\phi^{n+1}
\end{equation*}
Because $\phi^{n+1}$ satisfy a zero Neumann boundary condition hence the lagged pressure has a zero normal gradient.
\begin{equation}
\textbf{n}\cdot \nabla p^{n-1/2} = \textbf{n}\cdot Q(n)\nabla \phi^{n+1}
\end{equation}

This seems like to be a contradictory observation as the normal pressure gradient is not zero at n+1/2 step as indicated by the new update formula. Hence it seems that this is enforcing the newly calculated pressure gradient to be zero after each iteration. Therefore we infer that this choice of $q$ would result in a poorer approximation to the true normal pressure gradient especially where the analytical counterpart is not zero along the boundary. In practice, the normal pressure approximation is made consistent across all iterations because we don't force a change on $p^{n-1/2}$. However this implies the relation $q = p^{n-1/2}= Q(n)\phi^{n+1}$ does not hold any more. Hence the numerical boundary layer actually would not be filter out completely with this choice of $q$. Later we will see an illustration of this problem. Alg 3 does not suffer from this problem as $Q = 0$ (no pressure approximation is used at all). Therefore the normal mode analysis indicates appropriate choice of pressure approximation is critical to the accuracy of projection methods. A modification of $q$ for Alg 2 is proposed and discussed in the results section.\\

Now let's solve for the velocities.\\
First substitute the compatibility equation (5.11 (b)) into the momentum equation (5.11 (a)) to eliminate $\textbf{u}^*$ and then taking Laplace and Fourier transform in y to the resulting equation.\\
Also recall $q = Q(n)\phi^{n+1}$, $\hat{u}^{n+1} = z \hat{u}^n$, $\hat{\phi}^{n+1} = z \hat{\phi}^n$ and dropping the index $n+1$ we obtain \\

\begin{equation}
(\partial_x^2 - \bar{\mu}^2) \hat{\textbf{u}} = \dfrac{z \Delta t}{z + 1} [- (\partial_x^2 - \lambda^2) + \dfrac{2 Re \, Q(z)}{\Delta t}] \nabla \hat{\phi}
\end{equation}

where again we have defined $\bar{\mu}$ to be the positive real part of $\bar{\mu}^2 = k^2 + R \, \rho$, $\rho = \dfrac{2(z - 1)}{\Delta t (z + 1)}$ and $\lambda^2 = k^2 + \dfrac{2 R}{\Delta t}$ \cite{brown2001accurate}.\\

With the solution of $\hat{\phi}$ obtained earlier (equation 5.17) we can then solve for velocities accordingly
\begin{equation}
(\partial_x^2 - \bar{\mu}^2) \hat{\textbf{u}} = \dfrac{z \Delta t}{z + 1} [- (\partial_x^2 - \lambda^2) + \dfrac{2 R \, Q(z)}{\Delta t}] \nabla \,(\hat{\phi}_h + \hat{\phi}_p)
\end{equation}
Expanding the right hand side and using the definition of $\gamma^2$ and $\lambda^2$:
\begin{equation*}
\dfrac{z \Delta t}{z + 1} [- (\partial_x^2 - k^2) \nabla \hat{\phi}_h + \dfrac{2R}{\Delta t}(1+Q(z)) \nabla \hat{\phi}_h - \partial_x^2\,\nabla \hat{\phi}_p + \gamma^2\,\nabla \hat{\phi}_p] 
\end{equation*}
It is worth to note that most of the terms above can actually be dropped out including $\hat{\phi}_p$ which contains the spurious mode.\\
Too see this recall $(\partial_x^2 - k^2)\hat{\phi} = \hat{\eta}$ and hence the gradient of $\hat{\phi}$ satisfies the equation too by an interchange of operators. Therefore $(\partial_x^2 - k^2) \nabla \hat{\phi}_h =0$; now recall the particular solution of $\hat{\phi}$ is: $A_2 \,e^{-\gamma x}$ hence we have:
\begin{equation*}
- \partial_x^2\,\nabla \hat{\phi}_p + \gamma^2\,\nabla \hat{\phi}_p = 
\begin{cases}
-\partial_x^2\,\partial_x \,(A_2 \,e^{-\gamma x}) + \gamma^2\,\partial_x\,(A_2 \,e^{-\gamma x})
= \gamma^3\,A_2 \,e^{-\gamma x} - \gamma^3\,A_2 \,e^{-\gamma x} = 0\\
-\partial_x^2\,ij \,(A_2 \,e^{-\gamma x}) + \gamma^2\,ik\,(A_2 \,e^{-\gamma x})
= -\gamma^2 \,ik\,A_2 \,e^{-\gamma x} + \gamma^2\,ik\,A_2 \,e^{-\gamma x} = 0
\end{cases}
\end{equation*}

These results indicates that the spurious mode contained in $\hat{\phi}_p$ is now eliminated from the right hand side of equation 5.29, leaving a simpler equation to solve for $\hat{\textbf{u}}$:
\begin{equation}
(\partial_x^2 - \bar{\mu}^2) \hat{\textbf{u}} = \dfrac{2\, z (1+ Q(z))}{z + 1} \nabla \hat{\phi}_h
\end{equation}

For $\hat{u}$ component:
\begin{equation*}
(\partial_x^2 - \bar{\mu}^2) \hat{u} = - \dfrac{2 \, z (1+ Q(z))}{z + 1} \, |k| A_1 e^{- |k| x}
\end{equation*}
Solving this inhomogeneous ordinary differential equation we obtain
\begin{dmath*}
\hat{u} = U e^{-\bar{\mu} x} + \dfrac{2z\,F}{(z+1)\rho} |k| A_1 e^{- |k| x}
\end{dmath*}

Similar process can be used to solve for $\hat{v}$.\\

In summary we have:
\begin{equation}
\begin{cases}
\hat{u}_{nu} = U e^{-\bar{\mu} x} + \dfrac{R(z)}{\rho} \, |k| A_1 e^{- |k| x} \\
\hat{v}_{nu} = V e^{-\bar{\mu} x} - \dfrac{R(z)}{\rho} \, i k A_1 e^{- |k| x} \\
\hat{\phi}_{nu} = A_1 e^{- |k| x} + A_2 e^{- \gamma x} \\
\end{cases}
\end{equation}\\
where ``nu" refers to numerical solutions.\\

Now we can apply the boundary conditions to solve for the undetermined coefficients.\\
Recall at $x = 0$
\begin{dgroup}
\begin{equation}
\hat{u} |_{x = 0} = \hat{\alpha}, \, \, \, \hat{v} |_{x = 0} = \hat{\beta}, \, \, \, (\partial_x \hat{u} + ik \hat{v}) |_{x = 0} = 0
\end{equation}
\intertext{Additionally, as implied by the projection we have:\\

}
\begin{dmath}
\phi_x^{n+1} |_{x = 0} = 0 \condition{   $v^{n+1} |_{x = 0} = (v^* - \Delta t \phi_y^{n+1}) |_{x = 0} = \beta$}
\end{dmath}
\intertext{\\
Since we don't have access to $\phi^{n+1}$ (this becomes critically important for pressure free projection method) hence approximation is needed. Introducing function $\mathcal{B} (\phi)$ which approximates $\phi^{n+1}$ ($B$ depends on the particular projection methods). 3 choices are considered: $\mathcal{B} = 0, \phi^n$ and $2\phi^n - \phi^{n-1}$. Each corresponds to zeroth order, first order and second order approximations to $\phi^{n+1}$. We later see how this affects the accuracy of the methods.\\

After Fourier transformation we obtain\\
}
\begin{dmath}
\partial_x \hat{\phi} |_{x = 0} = 0
\end{dmath}
\intertext{introducing a Fourier transformed function $\hat{B} (z)$ such that}
\begin{dmath*}
\hat{v}^{n+1} |_{x = 0} = (\hat{v}^* - ik \Delta t \hat{\phi}^{n+1}) |_{x = 0} = \hat{\beta}
\end{dmath*}
\begin{dmath*}
\hat{v}^* = (\hat{v}^{n+1} + ik \Delta t \hat{B}\hat{\phi}) |_{x = 0}
\end{dmath*}
\intertext{Combine these 2 equations we obtain}
\begin{dmath}
(\hat{v}^{n+1}  + ik \Delta t (\hat{B} - 1) \hat{\phi} )|_{x = 0} = \hat{\beta}
\end{dmath}
\end{dgroup}
Hence we have the relation $\hat{\mathcal{B}} (\phi) = (\hat{B} - 1) \hat{\phi}$\\

Boundary condition of projection plays a critical role in the accuracy especially for pressure. It must be carefully chosen to ensure the boundary condition for velocity variables ($\textbf{u}^{n+1}$) satisfied under the context of the physics problem.\\

Recall from projection 
\begin{equation*}
\textbf{u}^* = \textbf{u}^{n+1} + \Delta t \nabla \phi^{n+1}
\end{equation*}

We have two choices. Recall the intermediate velocity ($\textbf{u}^*$) is an approximation to the true velocity at a time step in between $n+1$ and $n$ (The accuracy is however varied). With this in mind, we can simply choose
\begin{equation*}
\textbf{u}^{n+1} \,|_{x = 0} = \textbf{u}^* \,|_{x = 0} = \textbf{u} ((n+1) \Delta t) |_{x=0} = (\alpha, \, \beta)
\end{equation*}
Thus the physical boundary condition for $\textbf{u}^{n+1}$ is automatically satisfied. Further this naturally leads to a zero Neumann boundary condition for $\phi^{n+1}$ which is easier to implement. This is often used in Alg 1,2 because in this case, $\textbf{u}^*$ is set to be a good approximation to $\textbf{u}^{n+1}$ \cite{brown2001accurate,strikwerda1999accuracy}. However this also introduce problems in practice. Because we are imposing a Dirichlet boundary condition for $\textbf{u}^*$ and thus ignoring the coupling (interaction) between the boundary and interior notes. This could causing non-smoothness to develop along the boundary and thus increasing the error when conducting explicit differencing between boundary and nearby interior points \cite{brown2001accurate}. Another choice is leaving the boundary for intermediate velocity as computed, but then $\Delta t \phi^{n+1}$ is needed to compensate the difference.  This is often used in Alg 3 where $\textbf{u}^*$ is a less accurate approximation to $\textbf{u}^{n+1}$ \cite{kim1985application,brown2001accurate,strikwerda1999accuracy}. The problem with this choice is we simply don't have access to $\phi^{n+1}$ when solving for $\textbf{u}^*$ (this is also a coupled problem). Hence the boundary condition for $\textbf{u}^{n+1}$ is not exactly satisfied (but to an order of accuracy) \cite{strikwerda1999accuracy}. The strategy is to use ``good" approximation to $\phi^{n+1}$ and this is when equation (5.32 d) comes into play. Usually for Alg 3 a combination of the strategies is used resulting in the following boundary condition:
\begin{equation*}
\textbf{n} \cdot \textbf{u}^* = \textbf{n} \cdot \textbf{u}^{n+1}, \text{ Normal and } \, \, \, \textbf{$\tau$} \cdot \textbf{u}^* = \textbf{$\tau$} \textbf{u}^{n+1} + \mathcal{B} (\phi) \text{ Tangential}
\end{equation*}

In addition, we have shown that using the first strategy simplifies the numerical implementation and theoretically will not degrade the order of accuracy in pressure even for Alg 3. This is supported by the numerical results in \emph{Brown}'s paper where they have found that using accurate approximation to $\phi^{n+1}$ along boundary decreases the size of the error but do not affect the convergence rate \cite{brown2001accurate}.\\

Now let's compute the coefficients first\\
From Equation 5.32 a and the expression of $\hat{u}$ we obtain
\begin{equation}
U = \hat{\alpha} - \dfrac{R(z) \, |k|}{\rho} A_1
\end{equation}
From $\partial_x \hat{u} + ik \hat{v} = 0 |_{x = 0}$ we have
\begin{equation}
\bar{\mu} U = ik V
\end{equation}
From $\hat{\phi}_x = 0$  we have
\begin{equation}
-|k| A_1 - \gamma A_2 = 0 \, \, \, \Rightarrow A_2 = - \dfrac{|k|}{\gamma} A_1
\end{equation}

Combining E (5.32 (d)) with the expression of $\hat{v}$ we have
\begin{equation*}
V - \dfrac{ik \, R(z)}{\rho} A_1 + ik \Delta t (\hat{B} - 1) (A_1 + A_2) = \hat{\beta}
\end{equation*}

We can then solve for $A_1$ by using Equation 5.35 and the expressions for $U$ and $V$ and after some tedious algebra we obtain:
\begin{equation}
A_1 = E^{-1} (\bar{\mu} \hat{\alpha} - ik \hat{\beta})
\end{equation}
where we have defined: $E =  \dfrac{R \, R(z) |k|}{\bar{\mu} + |k|}(1 + \dfrac{C \, F(\hat{B} - 1)}{R(z)})$ and $C = \dfrac{2 |k|(\bar{\mu} + |k|)}{\gamma (\gamma + |k|)}$ for convenience.\\

$E$ represents precisely the coupling between the choice of pressure approximation ($R(z)$ and $F$) and choice of boundary condition of projection (this will affect $\hat{B}$). The coupling between these functions must be carefully chosen to maintain second order accuracy for all variables across the domain. We will see in the subsequent discussions how this can be met.\\

Then substitute the expression for $A_1$ back into Equation 5.33, 5.34 and 5.35 and we recover other coefficients:\\
In summary:
\begin{equation}
\begin{cases}
A_1 = \dfrac{(\bar{\mu} + |k|)\,(\bar{\mu} \hat{\alpha} - ik \hat{\beta})}{Re \, R(z) |k|}(1 + \dfrac{C \, F(\hat{B} - 1)}{R(z)})^{-1}\\
U = \hat{\alpha} - \dfrac{(\bar{\mu} \hat{\alpha} - ik \hat{\beta}) \, (\bar{\mu} + |k|)}{Re \, \rho} (1 + \dfrac{C\,F(\hat{B} - 1)}{R(z)})^{-1}\\
A_2 = - \dfrac{(\bar{\mu} \hat{\alpha} - ik \hat{\beta}) \, (\bar{\mu} + |k|)}{\gamma \, Re \rho R(z)}(1 + \dfrac{C\,F(\hat{B} - 1)}{R(z)})^{-1}\\
V = \dfrac{\bar{\mu} \hat{\alpha}}{i k} - \dfrac{\bar{\mu}(\bar{\mu} \hat{\alpha} - ik \hat{\beta}) \, (\bar{\mu} + |k|)}{ik \, Re \, \rho} (1 + \dfrac{C\,F(\hat{B} - 1)}{R(z)})^{-1}\\
\end{cases}
\end{equation}
%U = \hat{\alpha} - \dfrac{R(z) |k|}{\rho} E^{-1}\,(\bar{\mu} \hat{\alpha} - ik \hat{\beta})
%= \hat{\alpha} - \dfrac{(\bar{\mu} \hat{\alpha} - ik \hat{\beta}) \, (\bar{\mu} + |k|)}{Re \, \rho} (1 + %\dfrac{C\,F(\hat{B} - 1)}{R(z)})^{-1}

For our purpose of testing the accuracy, we want to compare the expression between the reference solutions (Equation 5.9) and our numerical solutions (Equation 5.31).\\

Let's first compare the coefficients of the second component solution in the $u$ velocity because both the numerical and the reference $\hat{u}$ solutions have the same decaying rate ($- |k|$).\\

Thus we compare
\begin{equation*}
\dfrac{R(z) |k|}{\rho} A_1 \text{ numericcal with } \, \, \, \dfrac{(\mu + |k|)}{Re \, s} (\mu \hat{\alpha} - ik \hat{\beta}) \text{ reference solution}
\end{equation*}
Followed from the result for $A_1$
\begin{equation*}
\dfrac{R(z) |k|}{\rho} A_1 = \dfrac{R(z) |k|}{\rho} \, E^{-1} (\bar{\mu} \hat{\alpha} - ik \hat{\beta})
\end{equation*}

Therefore it is obvious that we want 
\begin{equation}
\dfrac{R(z) |k|}{\rho} \, E^{-1} \approx \dfrac{(\mu + |k|)}{Re \, s} \, \text{ and } \, (\bar{\mu} \hat{\alpha} - ik \hat{\beta}) \approx (\mu \hat{\alpha} - ik \hat{\beta})
\end{equation}
in order to achieve optimal accuracy.\\

Let's do the second one first since it is easier! Keep in mind that it is our purpose to achieve second order accuracy in time for the numerical solutions. Hence quantitatively we want
\begin{equation*}
\mu \hat{\alpha} - ik \hat{\beta} = \bar{\mu} \hat{\alpha} - ik \hat{\beta} + \mathcal{O}(\Delta t^2)
\end{equation*}
where we have used the ``Big $\mathcal{O}$" notation to express the error.\\
Observe that the only term inhibits the accuracy is $\bar{\mu}$. \\

Recall $\bar{\mu}^2 = k^2 + Re \, \rho$ and  $\mu^2 = k^2 + Re \, s$
Hence it is $\rho$ which produces the error. A rough estimate indicate that we want $\rho$ converging to $s$ at least with second order accuracy. Hence let's prove this hypothesis first!\\

Recall the definition of $\rho$:
\begin{dgroup}
\begin{dmath*}
\rho = \dfrac{2(z-1)}{\Delta t (z+1)}
\end{dmath*}
\intertext{\\
because by definition, $z = e^{s\Delta t} = \sum_{n = 0}^\infty \dfrac{(s \Delta t)^n}{n !} = 1 + \mathcal{O}(\Delta t)$, hence $(z-1)^n = \mathcal{O}(\Delta t^n)$
\\
By Taylor expansion of $f(z) = \dfrac{z-1}{z+1}$ at $z = 1$ to order $(z-1)^3$ we obtain:
\\}
\begin{dmath}
f(z) = \sum_{n=0}^\infty \dfrac{f^{(n)}(1)\,(z-1)^n}{n!}
= f(1) + f'(1)\,(z-1) + \dfrac{f''(1)\,(z-1)^2}{2} + \mathcal{O}(\Delta t^3)
= \dfrac{e^{s\Delta t} - 1}{2} - \dfrac{(e^{s\Delta t} - 1)^2}{4}
\end{dmath}
\intertext{\\
By expanding $e^{s\Delta t}$ using the definition we find\\}
\begin{dmath}
f(z) = \dfrac{s\Delta t}{2} - \dfrac{(s\Delta t)^3}{4} + \mathcal{O}(\Delta t^3)
\end{dmath}
\intertext{\\
Then it is easy to show to that
\\}
\begin{dmath}
\rho = \dfrac{2f(z)}{\Delta t}
= s - \dfrac{s^3\Delta t^2}{2} + \mathcal{O}(\Delta t^2)
= s + \mathcal{O}(\Delta t^2)
\end{dmath}
\end{dgroup}

As for the error between $\bar{\mu}$ and $\mu$, it is straightforward to use a similar Taylor series argument and the result proven for $\rho$ to show 
\begin{equation}
\bar{\mu} = \mu + \mathcal{O}(\Delta t^2)
\end{equation}

These are the two important relations that we use in the accuracy test.\\

Now for the first comparison (in (4. 26)), there are three varying functions $C$ and $B(z)$ that we must consider to obtain second order accuracy.
\begin{equation*}
R(z) |k|E^{-1} = \dfrac{\bar{\mu} + |k|}{Re}(1 + \dfrac{C \, F(\hat{B} - 1)}{R(z)})^{-1} \text{ compare with } \dfrac{\mu + |k|}{Re} 
\end{equation*}

Hence according to the above formulation, we want $(1 + \dfrac{C \, F(\hat{B} - 1)}{R(z)})$ to converge to 1 and equivalently $\dfrac{C \, F(\hat{B} - 1)}{R(z)}$ to 0. By varying the pressure approximation function we can make this possible. Recall the 3 choices of $\mathcal{B}$ for projection methods
\begin{equation}
\begin{array}{lcl}
\mathcal{B} = 0 & \Rightarrow & \hat{B} = 1 \, \, \, \text{      Alg 1 and Alg 2}\\

\mathcal{B} = \phi^n & \Rightarrow & \hat{B} = 1/z \, \, \, \text{      Alg 3}\\

\mathcal{B} = 2\phi^n - \phi^{n-1} & \Rightarrow & \hat{B} = \dfrac{2}{z} - \dfrac{1}{z^2} \, \, \, \text{         Alg 3}\\
\end{array}
\end{equation}

Let's consider these methods separately here:\\

If $\mathcal{B} = 0$ as in Alg 1 and Alg 2, then we don't suffer from the error caused by $(1 + \dfrac{C \, F(\hat{B} - 1)}{R(z)})^{-1}$. Hence $R(z) |k|E^{-1}$ is a second order approximation to $\dfrac{\mu + |k|}{Re} $ since we have just proved $\bar{\mu} = \mu + \mathcal{O} (\Delta t^2)$. However it is important to note that enforcing a Dirchilet boundary condition might causing non-smoothness in spatial interpolation. Hence limiting the accuracy too \cite{strikwerda1999accuracy}. This is particular the case for Alg 3.\\

For Alg 3, the problem is more complicated. 
\begin{equation}
|\dfrac{C \, F(\hat{B} - 1)}{R(z)}| \leq |\dfrac{C \, F}{R(z)}| \, | \hat{B} - 1|
\end{equation}
We desire that the right hand side of the inequality vanishes as $\Delta t \rightarrow 0$. Luckily this can be achieved. If we let $\mathcal{B} = \phi^n$, then
\begin{dmath*}
\hat{B} - 1 = \dfrac{z - 1}{z}
\end{dmath*}
By using a Taylor expansion argument we can show that $\hat{B} = 1+ \mathcal{O}(\Delta t)$\\

Similarly for $\mathcal{B} = 2 \phi^n - \phi^{n-1}$ which is a second order approximation to $\phi^{n+1}$ would result in $\hat{B} - 1 = \mathcal{O} (\Delta t^2)$, however first order accurate is suffices here.\\

%k^2 + \dfrac{2Re\,F}{\Delta t}
Now for the first term in Equation 5.43, use the identity $| \gamma (\gamma + |k|) \, R(z) | \geqslant |\gamma^2|$
\begin{dmath*}
|\dfrac{C \, F}{R(z)}| = | \dfrac{2|k| (\bar{\mu} + |k|) \, F}{\gamma(\gamma + |k|) \, R(z)} |
\leq | \dfrac{2|k| (\bar{\mu} + |k|) \, F}{\gamma^2 \, R(z)} |
\end{dmath*}

Further recall $R(z) = \dfrac{2 z(1 + Q(z))}{1 + z}$ where $\hat{q} = Q(z)\hat{\phi}$ but $q = 0 \Rightarrow Q(z) = 0$. Therefore by Taylor expanding $R(z)$ at $z = 1$ again we find that
\begin{equation}
R(z) = \dfrac{2 z}{1 + z} = 1 + \dfrac{1}{2} s \Delta t
\end{equation}
Hence $R(z) = 1+\mathcal{O} (\Delta t)$.\\

$\Rightarrow$
\begin{equation}
| \dfrac{2|k| (\bar{\mu} + |k|) \, F}{\gamma^2 \, R(z)} | \leq | \dfrac{2|k| (\bar{\mu} + |k|) \, F}{\gamma^2}|
\end{equation}
\begin{equation*}
= |\dfrac{2|k| (\bar{\mu} + |k|) \, F}{k^2 + \dfrac{2 Re \, F}{\Delta t}}| 
\leq |\dfrac{2|k| (\bar{\mu} + |k|) \, F}{\dfrac{2 Re \, F}{\Delta t}}| 
\end{equation*}
$\Rightarrow$
\begin{dmath*}
|\dfrac{C \, F}{R(z)}| \leq |\dfrac{\Delta t |k| (\bar{\mu} + |k|)}{Re}| \, \, \, (\mathcal{O} (\Delta t))
\end{dmath*}

Hence together we have 
\begin{equation}
|\dfrac{C \, F (\hat{B} - 1)}{R(z)}| = \mathcal{O} (\Delta t^2)
\end{equation}
This leads to 
\begin{equation*}
\dfrac{R(z) |k|}{\rho} A_1 = \dfrac{(\mu + |k|)}{Re \, s} (\mu \hat{\alpha} - ik \hat{\beta}) + \mathcal{O} (\Delta t^2)
\end{equation*}
This relation holds for all projection methods.\\

Now we compute the accuracy for the first component of $\hat{u}$: compare 
$U e^{-\bar{\mu} x}$ numerical and $\dfrac{(\mu + |k|)}{Re \,s} (- |k| \hat{\alpha} + ik \hat{\beta}) e^{-\mu x}$ reference solution.\\

First note since $\bar{\mu} = \mu + \mathcal{O} (\Delta t^2)$ then
\begin{equation*}
| e^{\bar{\mu}x} - e^{\mu x} | = | \sum_{n=0}^{\infty} \dfrac{(\bar{\mu} x)^n - (\mu x)^n}{n!} |
\end{equation*}
\begin{equation*}
= | 0 + (\bar{\mu}x - \mu x) + \dfrac{((\bar{\mu} x)^2 - (\mu x)^2)}{2} + \cdots
\end{equation*}
\begin{equation*}
\leq |\bar{\mu} - \mu| |x| + |\dfrac{((\bar{\mu} x)^2 - (\mu x)^2)}{2}| + \cdots
\end{equation*}
Neglecting higher error terms we obtain
\begin{equation*}
| e^{\bar{\mu}x} - e^{\mu x} | \leq \mathcal{O} (\Delta t^2)
\end{equation*}

As for their coefficients\\
\begin{equation*}
U = \hat{\alpha} - \dfrac{(\bar{\mu} \hat{\alpha} - ik \hat{\beta}) \, (\bar{\mu} + |k|)}{Re \, \rho} (1 + \dfrac{C\,F(\hat{B} - 1)}{R(z)})^{-1}
= \dfrac{(\bar{\mu} + |k|) (-|k| \hat{\alpha} + ik \hat{\beta})}{Re \rho} (1 + \dfrac{C\,F(\hat{B} - 1)}{R(z)})^{-1}
\end{equation*}
Because $(1 + \dfrac{C\,F(\hat{B} - 1)}{R(z)})^{-1} = 1 + \mathcal{O} (\Delta t^2)$ , hence we can treat this term as $1$ and only need to compare the rest of the $U$ with the reference solution coefficient.
\begin{equation*}
| \dfrac{(\bar{\mu} + |k|) (-|k| \hat{\alpha} + ik \hat{\beta})}{Re \rho} - \dfrac{(\mu + |k|) (-|k| \hat{\alpha} + ik \hat{\beta})}{Re s} |
= | (\dfrac{(\bar{\mu} + |k|)}{\rho} - \dfrac{(\mu + |k|)}{s}) \, \dfrac{(-|k| \hat{\alpha} + ik \hat{\beta})}{Re}|
\end{equation*}
\begin{equation*}
\leq | (\dfrac{\mathcal{O} (\Delta t^2)}{(\mu - |k|)^2}| \, |(-|k| \hat{\alpha} + ik \hat{\beta})|
\end{equation*}

Combine these results we finally obtain an error bound for the $u$ velocity\\

$| \hat{u}_{numerical} - \hat{u}_{reference} | $
\begin{equation*}
\leq |U - \dfrac{(\mu + |k|) (-|k| \hat{\alpha} + ik \hat{\beta})}{Re s}| \, |e^{-\bar{\mu} x} - e^{-\mu x} | + | \dfrac{R(z) |k|}{\rho} A_1 - \dfrac{\mu + |k|)}{Re s} (\mu \hat{\alpha} - ik \hat{\beta}) | \, |e^{- |k|x}|
\end{equation*}
\begin{equation*}
= | \dfrac{(\bar{\mu} + |k|) (-|k| \hat{\alpha} + ik \hat{\beta})}{Re \rho} - \dfrac{(\mu + |k|) (-|k| \hat{\alpha} + ik \hat{\beta})}{Re s} |\, |e^{-\bar{\mu} x} - e^{-\mu x} | 
\end{equation*}
\begin{equation*}
+ | \dfrac{R(z) |k|}{\rho} E^{-1} (\mu \hat{\alpha} - ik \hat{\beta}) - \dfrac{\mu + |k|)}{Re s} (\mu \hat{\alpha} - ik \hat{\beta}) | \, |e^{- |k|x}|
\end{equation*}
\begin{equation*}
= | \mathcal{O} (\Delta t^2) |\cdot |\mathcal{O} (\Delta t^2)| + |\mathcal{O} (\Delta t^2)| \cdot| (\mu \hat{\alpha} - ik \hat{\beta}) | \cdot |e^{- |k|x}|
\leq \mathcal{O} (\Delta t^2) 
\end{equation*}

Hence overall $\hat{u}$ is second order accurate in time. Similar results shows $\hat{v}$ is second order accurate too. These results hold for all projection methods (Alg 1, Alg 2 and Alg 3).\\

The error analysis for pressure is more complicated since the accuracy depends on the particular projection methods.\\

From E (4.12) and E (4.13) we found that
\begin{equation*}
\hat{p}^{n+1/2} = Q(z)\hat{\phi}^{n+1} + L\hat{\phi}^{n+1}
\end{equation*}
Since $n + 1/2$ is half time step away from $n+1$, hence $\hat{p}^{n+1} = z^{1/2} \, \hat{p}^{n+1/2}$. Dropping the $n+1/2$ index we obtain an equation relating the pressure and the scalar potential ($\phi$) resulted from the projection\\
\begin{equation}
\hat{p} = z^{1/2} \, (Q(z) + L)\hat{\phi}
\end{equation}
This equation is the Fourier transformed Pressure update formula and hence governs the accuracy of pressure. The choice of $Q(z)$ and $L$ varies between projection methods and they must be chosen carefully to ensure second order accuracy.\\

Let's consider the 3 projection methods separately.\\
1. Alg 1\\
This corresponds to $q = p^{n-1/2}$, $L = I$ and $\mathcal{B} = 0 \, (\hat{B} = 1)$. The boundary condition for $\textbf{u}^{n+1}$ is satisfied exactly. Substituting this into the previous equations leads to
\begin{equation*}
\hat{q} = Q(z)\,\hat{\phi} = \hat{p}^{n-1/2} = \dfrac{1}{z^{3/2}}\,\hat{p} = \dfrac{1}{z^{3/2}} \, z^{1/2} (Q(z) + 1) \hat{\phi}
\end{equation*}
\begin{equation*}
\Rightarrow Q(z) = \dfrac{1}{z} (Q(z) + 1)
\end{equation*}
Hence we obtain a simple expression for $Q(z)$
\begin{equation}
Q(z) = \dfrac{1}{z-1}
\end{equation}
and therefore
\begin{equation}
\hat{p} = \dfrac{z^{3/2}}{z-1} \, \hat{\phi} = \dfrac{z^{3/2}}{z-1}\, A_1 e^{-|k|x} + \dfrac{z^{3/2}}{z-1} \, A_2 e^{-\gamma x}
\end{equation}
It is obvious that this formula would result in a degraded accuracy because it contains the spurious mode $A_2 e^{-\gamma x}$. The spurious mode is:
\begin{equation*}
- \, \dfrac{(\bar{\mu} \hat{\alpha} - ik \hat{\beta})\,(\bar{\mu} + |k|)}{\gamma Re \rho R(z)}
\end{equation*}

Because this term will not converge to zero and hence the pressure in Alg 1 will remain less than second order accurate. In fact, as proved in \emph{Brown et. al 2001} paper \cite{brown2001accurate}, $\dfrac{z^{3/2}}{z-1}\,A_2 \,e^{-\gamma x} \sim \mathcal{O} (\Delta t)$ and hence pressure will actually be first order accurate.\\
\textbf{I am not totally sure how Brown proved $\dfrac{z^{3/2}}{z-1}\,A_2 \,e^{-\gamma x} \sim \mathcal{O} (\Delta t)$. Maybe this is to do with $\gamma$? ($\gamma^2 = k^2 + \dfrac{2Re}{\Delta t}F$). But the degradation in accuracy is clear. }\\

2. Alg 2. The pressure approximation and boundary condition for $\textbf{u}^*$ are the same with Alg 1 however in this case we are using the more accurate pressure update formula. 

\begin{dmath}
\hat{p} = z^{1/2}\,(Q(z) + 1)\,A_1\,e^{-|k|x}
= \dfrac{z^{3/2}}{(z-1)} \dfrac{1}{R(z)} \dfrac{(\bar{\mu} + |k|)\,(\bar{\mu} \hat{\alpha}  -ik \hat{\beta})}{Re\,|k|}\,e^{-|k|x}
\end{dmath}

with $R(z) = \dfrac{2z(1+Q(z))}{1+z} = \dfrac{2z^2}{z^2 - 1}$ we obtain:\\

\begin{equation}
\hat{p} = \dfrac{z+1}{2z^{1/2}}\,\dfrac{(\bar{\mu} + |k|)\,(\bar{\mu} \hat{\alpha}  -ik \hat{\beta})}{Re\,|k|}\,e^{-|k|x}
\end{equation}

and then we know
\begin{equation*}
|\,\hat{p}_{nu} - \hat{p}_{ex}\,| =|\, \dfrac{z+1}{2 z^{1/2}} \dfrac{(\bar{\mu} + |k|)\,(\bar{\mu} \hat{\alpha}  -ik \hat{\beta})}{Re\,|k|} - \dfrac{(\mu + |k|)\,(\mu \hat{\alpha}  -ik \hat{\beta})}{Re\,|k|}\,|
\end{equation*}
By using a Taylor series expansion we find that $\dfrac{z+1}{2z^{1/2}} = 1 + \mathcal{O}(\Delta t^2)$ and combine the result that $\bar{\mu} = \mu + \mathcal{O}(\Delta t^2)$, it is straightforward to show that $\hat{p}_{nu} = \hat{p}_{ex} + \mathcal{O}(\Delta t^2)$

Hence here we have proved that by eliminating the spurious mode, the new pressure update formula used in Alg 2 indeed proves second order accuracy in pressure.\\

Does the same result hold for Alg 3 where the same formula for pressure is used? Well the answer is yes but we need to pay attention to the boundary conditions of the projection.\\

Since pressure is not involved in the update of solutions ($q = 0$) leads to $Q(z) = 0$. The transformed version is
\begin{equation}
\hat{p} = z^{1/2}\,\hat{\phi}_1 = z^{1/2}\,A_1\,e^{-|k|x}
\end{equation}
which again eliminates the spurious mode.\\

We want the coefficient $A_1$ be at least second order accurate compared to the coefficient for the reference pressure solution. As shown by our previous analysis, all the boundary conditions suffices here. However better approximation to $\phi^{n+1}$ leads to a decrease in size of error. More importantly as noted by many researchers including Brown and Strikwerda enforcing $\textbf{$\tau$}\cdot\phi^{n+1}$ to be zero causes non-smoothness for $\textbf{u}^*$ along the boundary because $\textbf{u}^*$ is not a second order approximation to $\textbf{u}^{n+1}$ now \cite{strikwerda1999accuracy}. This is supported by our numerical results in Chapter 6 section: ``Necessity for accurate approximation to $\phi^{n+1}$".\\

Hence we have demonstrated that the error bounds can be second order for all the projection methods considered provided the better pressure approximation formula is used.\\

\subsection{Normal mode analysis of Gauge method}
In this subsection, the accuracy of the Gauge method is explored. We will demonstrate that the introduction consistent update of Gauge variable lead to significant improvement over the accuracy in pressure simulations.\\

For simplicity let's take the periodic channel analysed for Projection methods before. The geometry is the same and hence we take the same boundary conditions for analytical velocities ($\textbf{u}$).\\
Recall the semi-discretised Gauge formulation for the linearised Stokes equations:
\begin{dgroup}
\begin{equation}
\dfrac{m^{n+1} -m^n}{\Delta t} = \dfrac{1}{2} \nabla^2\left(m^{n+1} + m^n\right)
\end{equation}
\intertext{\\
By the periodic geometry,the boundaries at $y = 0,\,2\pi$ are periodic. Only the Dirichlet boundary at $x=0$ needs to be specified. At this boundary, the x direction is normal and the y direction is tangential. Hence we have\\
}
\begin{dmath}
u(0,y,t) = \alpha(y,t) \condition{   and $v(0,y,t) = \beta(y,t)$}
\end{dmath}
\begin{dmath}
m_1(0,y,t) = \alpha(y,t) \condition{   and $m_2(0,y,t) = v(0,y,t) + \partial_y \chi (0,y,t)$}
\end{dmath}
\intertext{\\
Performing projection\\
}
\begin{dmath}
\nabla^2 \chi^{n+1} = \nabla \cdot \textbf{m}^{n+1} \condition{   with $\partial_x \chi^{n+1} = 0$}
\end{dmath}
\begin{dmath}
\textbf{u}^{n+1} = \textbf{m}^{n+1} - \nabla \chi^{n+1}
\end{dmath}
\intertext{\\
By making the normal boundary condition the same for the Gauge variable and velocity $\textbf{u}^*$, we obtain a zero Neumann boundary condition for the auxiliary field $\chi^{n+1}$. Similar to the projection method, this is one of the most common boundary condition for auxiliary fields.}
\end{dgroup}

Again the Reynolds number is made to be 1 for simplicity.\\
Because the Gauge variable $\textbf{m}$ is being updated consistently, hence like velocities and pressure we have a nice relationship between the Gauge variables calculated at different time iterations: $\textbf{m}^{n+1} = z \textbf{m}^n$ (where $z$ again is the discrete Laplace transform variable). This was not possible for the intermediate velocity field shown in the projection methods because it is rather discarded after each iteration. We cannot really locate the variable in time and thus makes the tracking of its errors difficult.\\

In this case, we can solve the Gauge variables directly by performing Laplace and Fourier transforms.\\
\begin{equation*}
\dfrac{\hat{\textbf{m}} - \dfrac{\hat{\textbf{m}}}{z}}{\Delta t} = \dfrac{1}{2}\left(\partial_x^2 - k^2 \right)
\left(\hat{\textbf{m}} + \dfrac{\hat{\textbf{m}}}{z} \right)
\end{equation*}
After rearranging and define $\bar{\mu} = k^2 + \rho$ where $\rho = \dfrac{2(z-1)}{\Delta t \,(z+1)}$, the transformed momentum equations (in components) are:

\begin{dgroup}
\begin{dmath}
\left( \partial^2_x - \bar{\mu}^2 \right) \, \hat{m}_1 = 0
\end{dmath}
\begin{dmath}
\left( \partial^2_x - \bar{\mu}^2 \right) \, \hat{m}_2 = 0
\end{dmath}
\end{dgroup}
Solving these 2 ordinary differential equations we obtain an expression for $\hat{textbf{m}}$:
\begin{dgroup}
\begin{dmath}
\hat{m}_1 = A_1\,e^{-\bar{\mu} x}
\end{dmath}
\begin{dmath}
\hat{m}_2 = A_2\,e^{-\bar{\mu} x}
\end{dmath}
\end{dgroup}
where $A_1$ and $A_2$ are constants to be determined.\\
Note that unlike the intermediate velocities in Projection methods, the Gauge variable actually does not contain any spurious mode! This would indeed lead to an improved accuracy since there is no numerical boundary in $\textbf{m}$.\\
This in turn guarantees the auxiliary field is also free of spurious modes. It is solved by the compatibility condition (\textbf{equation ()}):
\begin{equation}
\left(\partial_x^2 - k^2 \right)\,\hat{\chi} = \partial_x \hat{m}_1 + ik\hat{m}_2
\end{equation}
which then leads to an expression of $\hat{\chi}$ as:
\begin{dmath}
%\hat{\chi} = \dfrac{1}{\rho} \left(P\,e^{-|k|\,x} - \bar{\mu} \hat{m}_1 + ik\hat{m}_2 \right)
%= \dfrac{1}{\rho} \left(P\,e^{-|k|\,x} - A\,\bar{\mu} e^{-\bar{\mu} x} + ik\,B\,e^{-\bar{\mu} x} \right)
\hat{\chi} = P e^{-|k|x} + \dfrac{(-\bar{\mu}A_1 + ikA_2)}{\rho}e^{-\bar{\mu}x}
\end{dmath}

The velocities and pressure can then be recovered using the compatibility condition and the Gauge update formula respectively.\\

\begin{dgroup}
\begin{dmath}
\hat{u} = \hat{m}_1 - \partial_x \hat{\chi} = |k|Pe^{-|k|x} + \dfrac{\left(-k^2A_1 + ikA_2\bar{\mu} \right)}{\rho}e^{-\bar{\mu}x}
\end{dmath}
\begin{dmath}
\hat{v} = \hat{m}_2 - ik \hat{\chi} = -ikPe^{-|k|x} + \dfrac{\left(A_2\bar{\mu}^2 + ikA_1\bar{\mu} \right)}{\rho}e^{-\bar{\mu}x}
\end{dmath}
\end{dgroup}
The pressure can be recovered using the transformed Pressure update formula:
\begin{equation*}
\dfrac{\hat{p}}{\sqrt{z}} = \dfrac{1}{\Delta t}\left(1 - \dfrac{1}{z} \right)\,\hat{\chi} - \dfrac{1}{2} \left(\partial_x^2 - k^2\right)\left(1 + \dfrac{1}{z}\right)\,\hat{\chi}
\end{equation*}
Leading to 
\begin{equation}
\hat{p} = \dfrac{z+1}{2\,z^{1/2}}\left(\rho\,Pe^{-|k|x}\right)
\end{equation}
Unlike in projection method, the spurious mode must be eliminated with special pressure update formulas, the velocities and pressure in Gauge method naturally leads to the exclusion of spurious modes. Hence this demonstrates that no numerical boundary layers are formed in Gauge method. This is also confirmed by our numerical studies presented in Chapter 6.\\

Equipped with the boundary conditions given, we can then solve for the undetermined coefficients in the solutions above. \\

Normal component:
\begin{equation}
\hat{m}_1 \,|_{x=0} = \hat{u}\, |_{x=0} = \hat{\alpha}
\end{equation}

The tangential boundary condition for $\hat{textbf{m}}$ needs more attention, because by compatibility condition, solving $\textbf{m}^{n+1}$ requires the knowledge of $\chi^{n+1}$ which we don't have access yet:
\begin{equation}
\hat{u}\,|_{x=0} = \hat{m}_2\,|_{x=0} - ik \hat{\chi}^{n+1}\,|_{x=0} = \hat{\beta}\,\,\,\text{   exact relation}
\end{equation}
An approximation to $\hat{\chi}^{n+1}$ is needed and similar to the projection methods, we use the same function: $\hat{B}$ to represent the different approximations.\\
%\begin{equation*}
%\hat{B} = 1,\,\dfrac{1}{z},\,\dfrac{2z-1}{z^2}\,\,\, \text{   for 
%\end{equation*}
\begin{equation}
\hat{m}_2\,|_{x=0} = \hat{u}\,|_{x=0} + ik \hat{B}\hat{\chi}^{n+1}\,|_{x=0} \,\,\,\text{   actual boundary condition used in practice}
\end{equation}
Combine \textbf{equation () and ()} we get:
\begin{equation}
\hat{u}\,|_{x=0} + ik \hat{\chi}\,(\hat{B}-1)\,|_{x=0} = \hat{\beta}
\end{equation}

Same problem in regarding to the choice of auxiliary field approximation function $\hat{B}$. We can choose no approximation to $\chi^{n+1}$ which corresponds to $\hat{B} = 1$. This is artificially enforcing the Gauge variable and the velocities to be equal at all boundaries. This as noted by many researchers including \emph{strikwerda, Brown} that this would introduce non-smooth along boundary as the Gauge variable (or the intermediate velocity in Alg 3) is not be a close approximation to the velocities \cite{strikwerda1999accuracy, brown2001accurate}. This choice of tangential boundary condition for $\textbf{m}$ lead to degraded accuracy in velocities and pressure. There is a nice proof by Strikwerda in \cite{strikwerda1999accuracy} about the neccessity for tangential boundary conditions. Later we will demonstrate this through numerical tests.\\

Now because the tangential boundary condition for $v$ is not exactly satisfied in the projection step (see \textbf{equation ()}), we need to ensure it approaches $\beta$ along the boundary at a second order rate. Hence second order approximation to $\phi^{n+1}$ is needed and this corresponds to $\hat{B} = \dfrac{2z-1}{z^2}$ (For Alg 3 first order approximation to $\phi^{n+1}$ is sufficient since the approximation term is a multiplied with $\Delta t$ by the compatibility equation).\\

Hence along the boundary (taking $x=0$) we note the Gauge variables must satisfy:
\begin{dgroup}
\begin{dmath}
A_1 = \hat{\alpha} \condition{   $m_1$ at $x=0$}
\end{dmath}
\begin{dmath}
A_2 - ik\left(\dfrac{2z-1}{z^2}\right)\left(P + \dfrac{(-\bar{\mu}A_1 + ikA_2)}{\rho} \right) = \hat{\beta}
\end{dmath}
\end{dgroup}

Further because the auxiliary field satisfies a zero Neumann normal boundary condition ($\partial_x\hat{\chi} = 0$) hence we have:
\begin{equation}
\bar{\mu}^2A_1 - ik\bar{\mu}A_2 - |k|\rho\,P = 0
\end{equation}

Summaries these conditions we can solve for $A_1,\,A_2$ and $P$ directly:
\begin{equation}
\begin{bmatrix}
\bar{\mu}^2 & -ik\bar{\mu} & -|k|\rho \\
ik\bar{\mu}\hat{B}(z) & \left(\rho + k^2\hat{B}(z)\right) & -ik\hat{B}(z)\rho\\
1 & 0 & 0 \\
\end{bmatrix}
\begin{bmatrix}
A_1\\
A_2\\
P\\
\end{bmatrix}
= \begin{bmatrix}
0\\
\hat{\beta}\rho\\
\hat{\alpha}\\
\end{bmatrix}
\end{equation}
where $\hat{B}(z) = \dfrac{2z-1}{z^2}$. Further it can be shown by a Taylor expansion argument that $\dfrac{1}{\hat{B}(z)} = \dfrac{z^2}{2z-1} = 1 + \mathcal{O}(\Delta t^2)$. Substitute this back into the linear system above (Equation 5.78) and then we can solve for $A_1,\,A_2$ and $P$ easily.

\begin{dgroup}
\begin{dmath}
A_1 = \hat{\alpha}
\end{dmath}
\begin{dmath}
A_2 = \dfrac{\bar{\mu}^2\hat{\alpha} - (\bar{\mu} + |k|)(\bar{\mu}\hat{\alpha} - ik\hat{\beta})}{ik\bar{\mu}} + \mathcal{O}(\Delta t^2)
\end{dmath}
\begin{dmath}
P = \dfrac{(\bar{\mu} + |k|)(\bar{\mu}\hat{\alpha} - ik\hat{\beta})}{\rho \,|k|} + \mathcal{O}(\Delta t^2)
\end{dmath}
\end{dgroup}

The numerical solutions can then be recovered using Equation 5.70 and 5.71 by directly substituting $A_1,\,A_2$ and $P$ into the expressions for $\hat{u}, \,\hat{v}$ and $\hat{p}$.
\begin{equation}
\begin{cases}
\hat{u}_{nu} = \left(\dfrac{(\bar{\mu}+|k|)(-|k|\hat{\alpha}+ik\hat{\beta})}{\rho} + \mathcal{O}(\Delta t^2) \right)e^{-\bar{\mu}x} + \left(\dfrac{(\bar{\mu} + |k|)(\bar{\mu}\hat{\alpha} - ik\hat{\beta})}{\rho} + \mathcal{O}(\Delta t^2) \right)e^{-\bar{\mu}x}\\
\hat{v}_{nu} =  \left(\dfrac{i\bar{\mu}(\bar{\mu} + |k|)\,\left(- \dfrac{k}{|k|}\hat{\alpha}+i\hat{\beta}\right)}{\rho} +\mathcal{O}(\Delta t^2)\right) e^{-\bar{\mu} x} +  \left(\dfrac{-ik(\bar{\mu} + |k|)(\bar{\mu}\hat{\alpha} - ik\hat{\beta})}{\rho|k|} +\mathcal{O}(\Delta t^2)\right) e^{- |k| x} \\
\hat{p}_{nu} = \dfrac{z+1}{2z^{1/2}}\,\left(\dfrac{(\bar{\mu} + |k|)(\bar{\mu}\hat{\alpha} - ik\hat{\beta})}{|k|} + \mathcal{O}(\Delta t^2) \right)e^{-\bar{\mu}x}\\
\end{cases}
\end{equation}
where $nu$ denotes ``numerical" to distinguish from the analytical solutions in Equation 5.9.\\

Now recall from the subsection Projection methods, we have shown that $\rho = s + \mathcal{O} (\Delta t^2), \, \bar{\mu} = \mu + \mathcal{O} (\Delta t^2)$ and $\dfrac{z+1}{2z^{1/2}} = 1 + \mathcal{O} (\Delta t^2)$. Hence it is straightforward to show that the coefficients in the numerical solutions above are of second order accuracy to that of the analytical solutions presented in Equation 5.9.\\

Hence with the Gauge variable tangential boundary equals to $\textbf{$\tau$}\cdot\left(\textbf{u}^{n+1} + (2\chi^n - \chi^{n-1})\right)$, the Gauge method shows completely second order accuracy in both velocities and pressure. Also because neither the Gauge variable and the auxiliary field contains spurious modes, hence there is no special formula needed to filter out the numerical boundary layers as in Projection methods. This means the error convergence does not depend on the domain and boundary conditions as much as the projection methods which has fully second order accuracy only in periodic domains. The Gauge method exhibits the same order of accuracy in general domains too (see Shen' paper \cite{pyo2005normal,guermond2006overview} for details of proof).