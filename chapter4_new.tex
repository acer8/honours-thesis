\chapter{Error Analysis of Projection methods}
\label{chapter 4}
In this chapter we will present the error analysis for different ``second order" Projection methods and the Gauge method through rigorous normal mode analysis.

\section{Introduction of Normal mode analysis}

There are many ways to analyse the convergence and error for numerical methods, including the popular Energy methods \cite{liu1996projection,guermond2006overview} and Normal mode methods. Although the process of these methods differ but their essence is the same: compare the numerical solutions obtained to the true analytical solutions\cite{pyo2005normal,guermond2004error}. Energy methods can be used in a general setting but depends on the particular error structure of the projection methods and sometimes they are difficult to obtain \cite{guermond2006overview}. On the other hand, normal mode analysis is simple and reveals more precise information of the error of the projection methods. Hence recently there has been an increase of using normal mode analysis in the literature. However there is a major limitation because the analysis can only be performed on special domains and boundary conditions \cite{strikwerda1999accuracy,pyo2005normal,brown2001accurate}. For instance, in Brown. et. al 2001 paper \cite{brown2001accurate}, only one normal mode was considered and hence the results could not be easily generalised. Nevertheless, due to its simplicity and precision we will still perform a simple normal mode analysis to the ``second order" projection methods in this section.

\section{Normal mode analysis of Projection methods}
We are mainly concerned with the error in time stepping ($\Delta t$). The interaction of boundary conditions, pressure update and the second order implicit viscous terms are more important to us. Thus the advective term can be neglected. \cite{brown2001accurate,strikwerda1999accuracy,pyo2005normal,guermond2004error,liu1996projection,shen1996error,shen1992error}. This has been done in most of the textbooks and papers, and hence we will follow the convention and perform the analysis on the linearised unsteady Stokes equations.\\

We perform the analysis in a similar fashion to what has been done in David Brown paper \cite{brown2001accurate} and work with a periodic semi-infinite channel which is one of the simplest settings to consider slip boundary conditions.

\textbf{Periodic semi-infinite channel}\\
Errors in interior points can often easily be improved whereas it is more difficult to do so along the boundary. A physical boundary (e.g. no-slip boundary condition) could cause the error to behave differently. In most of the numerical studies (including this project) the flow is often confined in a bounded region, where slip-conditions are commonly used. For instance, a square domain with 4 sides. It is interesting to see how the Projection methods perform under these slip-boundary conditions. For the sake of simplicity we consider rather a microscopic picture of such domains where we focus on one side of the domain only. A Dirichlet boundary condition is imposed on it whereas other parts are left free. This is like a 1-dimensional flow in a 2-dimensional plane. We achieve this by using a semi-infinite periodic channel where the x direction flow is non-trivial but the y direction flow is fixed to be periodic in time. This is essentially the same as the one used in Brown's paper \cite{brown2001accurate}. While this geometry gives us an initial idea of the error behaviour for projection methods but it is too special and non-physical because we ignore the interaction between different boundaries. Whether error would behaves the same for other kinds of domains remains unanswered \cite{pyo2005normal}. \\

The domain is explicitly stated as:
\begin{equation*}
\Omega = \left[0, \infty \right) \times \left[0, 2\pi\right], \, \, \, t>0
\end{equation*}
and the boundary points satisfy:
\begin{equation*}\label{eq:set of boundary points}
\partial \Omega = \{(x,y) \in \mathbb{R}^2: x=0,\,y \in [0, 2\pi]\} \cup \{(x,y) \in \mathbb{R}^2: y = 0 \} \cup \{(x,y) \in \mathbb{R}^2: y = 2\pi \}
\end{equation*}

A Dirichlet boundary condition is imposed on the left end of the channel ($x = 0$) and the flow in $y$ is assumed to be equal to the top and bottom boundaries to ensure periodicity ($\textbf{u}(x,0,t) = \textbf{u}(x,2\pi,t)$).\\

Let's first compute the reference solution.\\
The 2 dimensional ($u(x, y, t), v(x, y, t), p(x, y, t)$) initial value problem of the linearised Stokes equations is presented below. We assume $u(x, y, t), v(x, y, t), p(x, y, t)$ are solutions to the Stokes equations and they satisfy both the momentum and continuity equations.

\begin{equation}\label{eq:linearised Stokes equations}
\begin{cases}
\textbf{u}_t + \nabla p = \dfrac{1}{R} \nabla^2 \textbf{u}\\
\nabla \cdot \textbf{u} = 0\\
\textbf{u} (x,y,t=0) = \textbf{u}_0 \condition{ initial value}\\
\end{cases}
\end{equation}

for $\forall (x,y) \in \Omega, \, t > 0$ and $R$ again denotes the Reynolds number\\
The problem is augmented with the following boundary conditions:
\begin{dmath*}
u(0,y,t) = \alpha (y,t)
\end{dmath*}
\begin{dmath*}
v(0,y,t) = \beta (y,t)
\end{dmath*}
for $\forall (x,y) \in \partial \Omega$ and $t > 0$\\

For simplicity, let's assume $\textbf{u}_0 = 0$ and also a constraint of vanishing velocity and pressure at infinity in the $x$ direction: $\textbf{u} = 0$ and $p = 0$ when $x \rightarrow \infty$ . By taking the divergence to the momentum equation given above, we arrive at another important condition for pressure:
\begin{dmath}
\nabla^2 p = 0 \condition{in $\Omega$}
\end{dmath}

Because of the periodic geometry, we are only concerned with the fluid motion in $x$ direction. Thus we regard the flow as a ``1-D" flow in $x$. We can then reduce the problem into ordinary differential equations (ODE) by taking Laplace transform in time and Fourier transform in the $y$ direction. Taking the transform in $x$ direction will yield the same result (of course this corresponds to periodic flow in $x$ instead). Let $s$ and $k$ denote the Laplace and Fourier transform variables respectively.

\begin{dgroup*}
\intertext{The unilateral Laplace transform is given as\\
}
\begin{dmath*}
\mathcal{L} [f(t)] = \int_0^\infty f(t) e^{-st} dt
\end{dmath*}
\intertext{$f(t)$ is an integrable function defined over the interval $[0, \infty]$\\
}
\intertext{The standard Fourier transform (with angular frequency $k$) is defined as\\
}
\begin{dmath*}
\mathcal{F} [f(y)] = \dfrac{1}{\sqrt{2 \pi}}\int_{-\infty}^{\infty} f(y) e^{-iky} dy
\end{dmath*}
\end{dgroup*}

We take Laplace transform followed by Fourier transform to the linearised Stokes equations. Let a hat ($\hat{.}$) denote the final transformed variable.\\
By the common Fourier and Laplace transform identities, we obtained the following transformed quantities.\\
\begin{equation*} 
\textbf{u} \rightarrow \hat{\textbf{u}} = (\hat{u}, \hat{v}), \, \, \, p \rightarrow \hat{p}
\end{equation*}
\begin{equation*}
\textbf{u}_t \rightarrow s \hat{\textbf{u}}, \, \, \, \nabla \cdot \textbf{u} \rightarrow \partial_x \hat{u} + ik \hat{v}
\end{equation*}
\begin{equation*}
\nabla^2 \textbf{u} \rightarrow (\partial_x^2 -k^2)\hat{\textbf{u}}, \, \, \,  \nabla p \rightarrow (\partial_x p, ik p)^T
\end{equation*}

Hence the transformed Stokes equations are:
\begin{dgroup}\label{eq:Laplace and Fourier transformed linear Stokes equations}
\begin{dmath}\label{eq:Laplace and Fourier transformed linear Stokes equations P u}
(\partial_x^2 - \mu^2) \hat{u} = R \, \partial_x \hat{p}
\end{dmath}
\begin{dmath}\label{eq:Laplace and Fourier transformed linear Stokes equations p v}
(\partial_x^2 - \mu^2) \hat{v} = ik \, R \, \hat{p}
\end{dmath}
\begin{dmath}\label{eq:Laplace and Fourier transformed Laplace of pressure}
(\partial_x^2 -k^2) \hat{p} = 0
\end{dmath}
\begin{dmath}\label{eq:Laplace and Fourier transformed divergence of velocity}
\partial_x \hat{u} + ik \hat{v} = 0
\end{dmath}
\intertext{for $\forall (x,y) \in \Omega$ and $t > 0$}
\begin{dmath}\label{eq:Laplace and Fourier transformed u boundary condition}
\hat{u} (0,y,t) = \hat{\alpha}
\end{dmath}
\begin{dmath}\label{eq:Lapalce and Fourier transfomred v boundary condition}
\hat{v} (0,y,t) = \hat{\beta} 
\end{dmath}
\end{dgroup}
for $\forall (x,y) \in \partial \Omega$ and $t > 0$
%Thus we obtain: $\partial t -> s, \, \, \, \partial y -> ik$
where $\mu^2 \equiv k^2 + R s$ and we take $\mu$ to be the positive real part of the solution for simplicity \cite{brown2001accurate}.\\

Because \eqref{eq:Laplace and Fourier transformed linear Stokes equations} are ordinary differential equations, hence we can solve them easily using method of characteristics. Let's start from the pressure equation \eqref{eq:Laplace and Fourier transformed Laplace of pressure}.\\

Its characteristic equation is given as:
\begin{equation*}
r^2 - k^2 = 0
\end{equation*}
Solving it we obtained:
\begin{equation*}
r = \pm |k|
\end{equation*}
The by superposition principle we obtained a solution to this ODE:
\begin{equation*}
\hat{p} = P_1 e^{|k| x} + P_2 e^{- |k| x}
\end{equation*}

Hence there are 2 distinct modes of motion: exponential growth and decay. However because of our infinite half plane geometry, we would see the pressure (and also velocities) grow to infinity as $x \rightarrow \infty$. This is physically impossible and we don't want to see our solutions blow up in space! Therefore the exponential growth mode must be dropped, leaving the solution as:

\begin{equation}
\hat{p} = P e^{-|k| x}
\end{equation}
where $P$ is a constant amplitude to be determined.\\
Hence we are left with only one normal mode. In fact, this exponential decaying motion is also referred as ``quasi-normal mode" in literature \cite{pyo2005normal}. \\

Thus we have seen the limitation of normal mode analysis already because the normal modes strongly depends on the spatial domain of the problem.\\

Substitute the expression for $\hat{p}$ into \eqref{eq:Laplace and Fourier transformed linear Stokes equations P u} we obtain an equation for $\hat{u}$:
\begin{dgroup}
\begin{dmath}
\partial_x^2 \, \hat{u} - \mu^2 \, \hat{u} = -|k| \, R \, P e^{-|k| x}
\end{dmath}
\intertext{\\
this is again an (inhomogeneous) ordinary differential equation for $\hat{u}$! \\
Solving it we obtained:\\
}
\begin{dmath}
\hat{u} = U e^{-\mu x} + \dfrac{|k|}{s} P e^{-|k| x}
\end{dmath}
\end{dgroup}
where once again the exponentially growing mode ($e^{-\mu x}$) is dropped out.\\

Similarly by substituting the expression of pressure (Equation 4.3) into the ODE for $\hat{v}$ (Equation 5.2 b) we obtain the solution for $\hat{v}$:
\begin{equation}
\hat{v} = V e^{-\mu x} - \dfrac{ik}{s} P e^{-|k| x}
\end{equation}

And by the divergence constraint \eqref{eq:Laplace and Fourier transformed linear Stokes equations p v} we obtain another equation relating $\hat{u}$ and $\hat{v}$
\begin{dmath}
- \mu U + ik \, V = 0
\end{dmath}

Now with 3 equations corresponding to 3 unknowns, we can solve for the coefficients $U, V$ and $P$ by applying the divergence free and boundary conditions given in equations \eqref{eq:Laplace and Fourier transformed divergence of velocity}, \eqref{eq:Laplace and Fourier transformed u boundary condition} and \eqref{eq:Lapalce and Fourier transfomred v boundary condition}. The problem is organised neatly in the following matrix form:

\begin{equation}
\begin{bmatrix}
1 & 0 & |k| / s \\
0 & 1& - i k / s \\
-\mu & ik & 0 \\
\end{bmatrix}
\begin{bmatrix}
U\\
V\\
P\\
\end{bmatrix}
= \begin{bmatrix}
\hat{\alpha}\\
\hat{\beta}\\
0\\
\end{bmatrix}
\end{equation}

By solving the matrix problem we obtained the coefficients for the solution of ODEs \eqref{eq:Laplace and Fourier transformed linear Stokes equations} as:

\begin{dgroup}
\begin{dmath}
U = \dfrac{(\mu + |k|)}{R \, s} (- |k| \hat{\alpha} + ik \hat{\beta})
\end{dmath}
\begin{dmath}
V = \dfrac{-i (\mu + |k|)\mu}{R \, s} (-\dfrac{k}{|k|} \hat{\alpha} + i \hat{\beta})
\end{dmath}
\begin{dmath}
P = \dfrac{(\mu + |k|)}{R \, |k|} (\mu \hat{\alpha} - ik \hat{\beta})
\end{dmath}
\end{dgroup}

And finally we arrived at the reference solutions for our initial value problem of the (transformed) linearised Stokes equations: 

\begin{dgroup}\label{eq:Reference solution for transformed linearised Stokes equations}
\begin{dmath}\label{eq:Reference solution for transformed linearised Stokes equations u component}
\hat{u}_{ex} = \dfrac{(\mu + |k|)}{R \, s} (- |k| \hat{\alpha} + ik \hat{\beta})e^{-\mu x} + \dfrac{(\mu + |k|)}{R \, s} (\mu \hat{\alpha} - ik \hat{\beta}) e^{-|k| x}
\end{dmath}
\begin{dmath}\label{eq:Reference solution for transformed linearised Stokes equations v component}
\hat{v}_{ex} = \dfrac{-i (\mu + |k|)\mu}{R \, s} (-\dfrac{k}{|k|} \hat{\alpha} + i \hat{\beta}) e^{-\mu x} - \dfrac{i k (\mu + |k|)}{R \, s \, |k|} (\mu \hat{\alpha} - ik \hat{\beta}) e^{-|k| x}
\end{dmath}
\begin{dmath}\label{eq:Reference solution for transformed linearised Stokes equations p component}
\hat{p}_{ex} = \dfrac{(\mu + |k|)}{R \, |k|} (\mu \hat{\alpha} - ik \hat{\beta}) e^{-|k| x}
\end{dmath}
\end{dgroup}
where the subscript ``ex" represents exact solutions.\\

\subsection{Normal analysis for Algorithm 1, 2 and 3}
Now let's perform the above analysis to the numerical schemes described in the previous chapter. Note that the numerical schemes we work with are only discretised in time because we are mainly concerned in the error behaviour in time stepping rather than spatial stepping. Therefore we will need to use the discrete Laplace transform whereas keep using the continuous Fourier transform.\\

with a time stepping of $\Delta t$ the discrete Laplace transform (or Z transform) is defined as \cite{strikwerda1999accuracy, strikwerda1999accuracy, brown2001accurate, pyo2005normal}:
\begin{equation*}
X(z) = \Delta t \sum_{n=-\infty}^{n=\infty} z^{-n} \, x(n), \text{     $z = e^{s \Delta t} $ and $|z| > 1$}
\end{equation*}
Note $z$ is the discrete transform variable (not the $z$ - direction) and it is related to the continuous transform by: $z = e^{s \Delta t}$.\\

The Z transform converts a series of discrete time signals $x(n)$ into the complex frequency domain. Hence $X^{n+1}$ is merely $X^n$ shifted up by $z$. To see this define a new set of signals $w$ being basically the original signal $x(n)$ at $n+1$ step. Hence $w^n = x^{n-1}$. Then by applying 
\begin{dmath}
X^{n+1} = W^n
= \Delta t \sum_{n=-\infty}^{n=\infty}\,w^n\,z^{-n} =  \Delta t \sum_{n=-\infty}^{n=\infty}\,x^{n-1}\,z^{-n} 
=  \Delta t \sum_{n=-\infty}^{n=\infty}\,x^n\,z^{-(n+1)} = z \Delta t \sum_{n=-\infty}^{n=\infty}\,x^n\,z^{-n} 
= z X^n
\end{dmath}

As discussed before there are 3 important things that are crucial to the accuracy of projection methods:\\
One: the choice of pressure approximation ($q$); Two: boundary condition of the intermediate velocity field and Three: choice of the non-physical variable $\phi$ resulted from the projection. David et.al has argued that the coupling between these 3 things must be considered to achieve second order accuracy for all primitive variables \cite{brown2001accurate}. We will also show this in the subsequent analysis.\\

Let's consider a general second order projection method with centred finite difference used to approximate the time derivatives and 2nd order Crank-Nicholson method used to treat the implicit diffusive terms. This is just the linearised scheme considered in the previous chapter (see Equation \eqref{eq:general 2nd discretisation of momentum equation to sovle for u*} to \eqref{eg:general 2nd velocity update})\\

The algorithm is summarised below\\
\begin{equation}\label{eq:general 2nd projection method for linearised Stokes equations}
\begin{cases}
\dfrac{(\textbf{u}^* - \textbf{u}^n)}{\Delta t} + \nabla q = \dfrac{R}{2} \nabla^2 (\textbf{u}^* + \textbf{u})\,\,\,\text{   solve for $\textbf{u}^*$}\\
\textbf{u}^* \,|_{\partial \Omega}= \left(\textbf{u}^{n+1} + \Delta t \phi^{n+1}\right)|_{\partial \Omega}\,\,\,\text{   Compatibility condition or projection step}\\
p^{n+1/2} = q + L \phi^{n+1}\,\,\,\text{   Pressure update}
\end{cases}
\end{equation}
Where $q, \,L$, boundary conditions for projection and $\phi^{n+1}$ depends on the particular projection scheme.

\paragraph*{Let's first consider the relation between $q$ and $\phi^{n+1}$}.\\
In Projection methods, the intermediate velocity field and auxiliary field are approximations to the true velocity and pressure, although the degree of accuracy is dependent on different methods. Hence we expect a close relationship between $\phi$ and $p$. In fact, in order to eliminate the potential numerical boundary layers that might arise in computations, David et.al has proposed a simple definition of $\phi$ \cite{brown2001accurate}:
\begin{mydef}\label{definition of q and phi in normal mode analysis}
\begin{equation*}
q^n = Q(n) \phi^n
\end{equation*}
Apply Laplace and Fourier transform:
\begin{equation}\label{eq:relation between transfromed q and phi}
\hat{q}^n = Q(z) \hat{\phi}^n
\end{equation}
Where $Q(z)$ is a function of Z transform variable that describes the coupling between the choice of pressure approximation ($q$) and the auxiliary field ($\phi$). This function indeed varies for different variations of projection methods. We will see shortly that this limit the choice of $q$ too.
\end{mydef}
In this section, we consider the 3 widely used Projection methods discussed in the previous chapter.\\

\textbf{Alg 1} - \emph{Projection with Lagged Pressure term (first order update formula)}

As discussed in the previous chapter this method uses a first order pressure update formula:
\begin{equation*}
p^{n+1/2} = p^{n-1/2} + \phi^{n+1}
\end{equation*}
Hence this method corresponds to
\begin{equation*}
q = p^{n-1/2}, \, \, \, L = I \text{   Identity matrix   }
\end{equation*}
And boundary condition for projection:
\begin{equation*}
\textbf{u}^* |_{\partial \Omega} = \textbf{u}^{n+1} |_{\partial \Omega}
\end{equation*}\\

\textbf{Alg 2} - \emph{Projection with Lagged Pressure term (second order update formula)}\\
This corresponds to a second order pressure update formula used to improve the accuracy for Alg 1

\begin{equation*}
p^{n+1/2} = p^{n-1/2} + \nabla \phi^{n+1} - \dfrac{\Delta t}{2 R} \nabla^2 \phi^{n+1}
\end{equation*}
\begin{equation*}
q = p^{n-1/2}, \, \, \, L = I - \dfrac{\Delta t}{2 R} \nabla^2
\end{equation*}
And same boundary condition for projection\\

\textbf{Alg 3} - \emph{Pressure free projection}\\
First proposed by Kim and Moin. Pressure can be recovered using the same second order update formula as in Alg 2. \cite{kim1985application,brown2001accurate}
\begin{equation*}
q = 0, \, \, \, L = I - \dfrac{\Delta t}{2 R} \nabla^2
\end{equation*}
Boundary condition for projection:
\begin{equation*}
\textbf{n} \cdot \textbf{u}^* |_{\partial \Omega} = \textbf{n} \cdot \textbf{u}^{n+1} |_{\partial \Omega}, \, \, \, \textbf{$\tau$} \cdot \textbf{u}^* |_{\partial \Omega} = \textbf{$\tau$} \left(\textbf{u}^{n+1}+ \Delta t \textbf{$\tau$} \nabla \phi^{n+1}\right) |_{\partial \Omega}
\end{equation*}

\paragraph*{Now Let's perform the Normal Mode analysis}.\\
Taking divergence of the momentum equation in \eqref{eq:general 2nd projection method for linearised Stokes equations} we can eliminate the divergence free velocity field. Note that the Laplace of $\textbf{u}^n$ is also divergence free because $\nabla \cdot \nabla^2 \,\textbf{u}^n = \nabla^2 \,\nabla \cdot \textbf{u}^n = 0$. Recall the definition of $q$ in Definition \ref{definition of q and phi in normal mode analysis} we obtain an equation between $\textbf{u}^*$ and $\phi$.

\begin{equation}
\nabla \cdot \textbf{u}^* + \Delta t \,Q(n)\,\nabla^2 \phi^{n+1} = \dfrac{\Delta t}{2} \nabla^2 \,(\nabla \cdot \textbf{u}^*)
\end{equation}

Also taking divergence of the compatibility equation \eqref{eq:general 2nd projection method for linearised Stokes equations} and substitute into the equation above we obtained an equation in terms of $\nabla^2\phi^{n+1}$, denote this as $\eta$
\begin{equation*}
\eta + Q(n)\eta = \dfrac{\Delta t}{2}\, \nabla^2 \eta
\end{equation*}
The variable $\eta$ describes the coupling between the intermediate velocity field $\textbf{u}^*$ and the scalar potential $\phi$ resulted from the projection. Therefore even though velocity and pressure are now decoupled from projection, their coupling is now represented by the intermediate velocity and the auxiliary variable $\phi$. Hence it is important to understand the structure of $\eta$ in order to understand the error in velocity and pressure for these projection methods.\\

Similar to the reference solution derivation, we take Laplace and Fourier transform to the equation above. After rearranging we obtain:
\begin{equation}
\partial_x^2 \hat{\eta} - \hat{\eta} (k^2 + \dfrac{2}{\Delta t}\,(1+Q(z))) = 0
\end{equation}
Further define $\gamma^2 = k^2 + \dfrac{2}{\Delta t}\,F(z)$ where $F(z) = 1 + Q(z)$ for convenience and taking $\gamma$ to be the positive real part of the previous equation.\\

By solving the ODE for $\hat{\eta}$ we found
\begin{equation}
\hat{\eta} = A\,e^{-\gamma x}
\end{equation}
Note the non-physical exponential growing mode $e^{\gamma x}$ has already been dropped.\\

The decaying rate $\gamma$ causes trouble because the $\Delta t$ in the denominator of $\gamma$ would cause the mode to become ill-defined as $\Delta t$ approaching zero. This makes $A\,e^{-\gamma x}$ a spurious mode. \cite{brown2001accurate, strikwerda1999accuracy}. The numerical boundary layer and its effects on accuracy is well studied by the work of \emph{E and Liu et al} \cite{liu1996projection}. \\
It is worth noting that the spurious mode obtained by solving $\eta$ comes from the compatibility equation. Hence it is an inherent property for all projection methods. Also recall the definition of $\eta$, it is customary to think that the divergence of intermediate velocity field ($\nabla \cdot \textbf{u}^*$) and the auxiliary field ($\phi$) would contain the spurious mode too. We will shortly demonstrate this claim. Therefore it is of interest to devise a method which ensures the actual velocity ($\textbf{u}$) and pressure ($p$) does not contain it. \\

With the expression for $\eta$ we can now solve for $\phi$ based on the second order inhomogeneous ordinary differential equation below:
\begin{equation}
(\partial_x^2 - k^2)\hat{\phi} = \eta = A\,e^{-\gamma x}
\end{equation}
\begin{equation}\label{eq:solution of numerical phi}
\hat{\phi} = \hat{\phi}_h + \hat{\phi}_p = A_1\,e^{-|k|x} + A_2 \,e^{-\gamma x}
\end{equation}
Where the subscript $h$ and $p$ denotes homogeneous and particular solutions from the ODE respectively.\\
We observed that $\phi_h$ contains the actual piece of solution we want whereas $\phi_p$ is not because it contains the spurious mode. Thus we have demonstrated our claim that the auxiliary variable $\phi$ contains the spurious mode. However as we shall see later that we do want $\nabla \cdot \textbf{u}^*$ converges to zero as $\Delta t \rightarrow 0$ and then the homogeneous solution would dominate.\\
Hence we can now solve the intermediate velocity field by taking the divergence of the compatibility condition:
\begin{equation}
\nabla \cdot \hat{\textbf{u}^*} = \Delta t \,(\partial_x^2 - k^2)\hat{\phi} = \Delta t\,A_2 e^{-\gamma x}\,(\gamma^2 - k^2)
\end{equation}
Because $\gamma^2 - k^2 = F(z) = 1 + Q(z)$ and this term obviously would not be zero for any of the projection methods, hence we have demonstrated that the divergence of the intermediate velocity field also contains the spurious mode. These findings are consistent with Brown, strikwerda and Shen \cite{brown2001accurate, strikwerda1999accuracy}.\\

Now it is interesting to see how the spurious mode would affect the accuracy for velocity and pressure in the projection methods. \\

For Alg 1, the pressure update formula is:
\begin{equation}
p^{n+1/2} = q + \phi^{n+1} = (Q(n) + 1)\phi^{n+1}
\end{equation}
Where the lagged pressure approximation $q = p^{n-1/2}$ is used. The transformed version of the update formula is:\\
\begin{equation}\label{eq:transformed pressure update formula for Alg 1}
\hat{p}^{n+1/2} = (Q(z)+1)\hat{\phi}
\end{equation}
Then by recalling the relation between $q$ and $\phi$ \eqref{eq:relation between transfromed q and phi} we found that:
\begin{equation*}
\hat{q} = Q(z)\,\hat{\phi} = \hat{p}^{n-1/2} = \dfrac{1}{z^{3/2}}\,\hat{p} = \dfrac{1}{z^{3/2}} \, z^{1/2} (Q(z) + 1) \hat{\phi}
\end{equation*}
Then we obtain
\begin{equation*}
Q(z) = \dfrac{1}{z} (Q(z) + 1)
\end{equation*}
This finally leads to the following expression of $Q(z)$:
\begin{equation}
Q(z) = \dfrac{1}{z-1}
\end{equation}
Hence we found that the pressure is updated as:
\begin{dmath}
\hat{p}^{n+1/2} = \dfrac{1}{z-1}\,(A_1\,e^{-|k|x} + A_2 \,e^{-\gamma x}) + A_1\,e^{-|k|x} + A_2 \,e^{-\gamma x}
= \dfrac{z}{z-1}\,\left(A_1 e^{-|k|x} +  A_2 \,e^{-\gamma x}\right)
\end{dmath}
The clearly indicates the presence of spurious mode. Hence the the accuracy is strongly limited due to formation of numerical boundary layer. The boundary layer not only causing degradation in accuracy along the boundary but could also affect the nearby interior points too. In some domains (like a squared domain with Drichlet boundary condition imposed), the spurious mode would cause strong oscillations along the boundary.\\

In addition as the projection implies, the auxiliary field often satisfies a normal Neumann boundary condition :
\begin{equation*}
\textbf{n} \cdot \nabla \phi^{n+1} = \dfrac{\partial \phi^{n+1}}{\partial n} = 0
\end{equation*}
This is often referred as a non-physical boundary condition in the literature \cite{strikwerda1999accuracy, guermond2004error, brown2001accurate} for the reasoning discussed below.\\
Extracting the normal component of the gradient of pressure along the boundary \eqref{eq:Bell first order pressure update formula} and use the result of the above equation we found:
\begin{equation}
\textbf{n} \cdot \nabla \hat{p}^{n+1/2}\,|_{\partial \Omega} = \textbf{n} \cdot \nabla \hat{p}^{n-1/2}\,|_{\partial \Omega}
\end{equation}
This implies the normal pressure gradient is constant along the boundary for all time steps. This is in general not consistent with the actual boundary condition of pressure gradient. Hence this partly explains why numerical boundary layer exists.\\

The other update formula used for Alg 2 and 3 should show an improved accuracy.\\
The transformed version of the formula is:
\begin{equation}
\hat{p}^{n+1/2} = \dfrac{1}{z^{1/2}}\hat{p} = Q(z)\hat{\phi} + \hat{\phi} - \dfrac{1}{2}\hat{\nabla \cdot \textbf{u}^*}
\end{equation}
substitute the expression for $\hat{\nabla \cdot \textbf{u}^*}$ and $\hat{\phi}$ into the equation above we obtain:
\begin{equation}
\dfrac{1}{z^{1/2}}\hat{p} = Q(z)\,A_1\,e^{-|k|x} + Q(z)\,A_2 \,e^{-\gamma x} + A_1\,e^{-|k|x} + A_2 \,e^{-\gamma x} - (1+Q(z)) \,e^{-\gamma x}
\end{equation}
After rearranging we found that the spurious mode $e^{-\gamma x}$ is actually being cancelled out, leaving the pressure as:
\begin{equation}
\hat{p} = z^{1/2}\,(Q(z) + 1)\,A_1\,e^{-|k|x} 
\end{equation}
where the $Q(z)$ varies depending on $q$. It could be $\dfrac{1}{z-1}$ as in Alg 1 and Alg 2 or zero in Alg 3. However in either case, the spurious is filtered out from the pressure update equation. Hence from this point of view, this formula shows an improved accuracy.\\

However this analysis needs more attention for Alg 2, because the pressure approximation uses a lagged pressure value and recall the relation between $q$ and $\phi$ we find:
\begin{equation*}
p^{n-1/2} = Q(n)\phi^{n+1}
\end{equation*}
Because $\phi^{n+1}$ satisfy a zero Neumann boundary condition hence the lagged pressure has a zero normal gradient.
\begin{equation}
\textbf{n}\cdot \nabla p^{n-1/2} = \textbf{n}\cdot Q(n)\nabla \phi^{n+1}
\end{equation}

This seems like to be a contradictory observation as the normal pressure gradient is not zero at n+1/2 step as indicated by the new update formula. Hence it seems that this is enforcing the newly calculated pressure gradient to be zero after each iteration. Therefore we infer that this choice of $q$ would result in a poorer approximation to the true normal pressure gradient especially where the analytical counterpart is not zero along the boundary. In practice, the normal pressure approximation is made consistent across all iterations because we don't force a change on $p^{n-1/2}$. However this implies the relation $q = p^{n-1/2}= Q(n)\phi^{n+1}$ does not hold any more. Hence the numerical boundary layer actually would not be filter out completely with this choice of $q$. Later we will see an illustration of this problem. Alg 3 does not suffer from this problem as $Q = 0$ (no pressure approximation is used at all). Therefore the normal mode analysis indicates appropriate choice of pressure approximation is critical to the accuracy of projection methods. A modification of $q$ for Alg 2 is proposed and discussed in the results section.\\

\paragraph*{Now let's solve for the velocities}.\\
First substitute the compatibility equation into the momentum equation in \eqref{eq:general 2nd projection method for linearised Stokes equations} to eliminate $\textbf{u}^*$ and then taking Laplace and Fourier transform in $y$ to the resulting equation.\\
Also recall $q = Q(n)\phi^{n+1}$, $\hat{u}^{n+1} = z \hat{u}^n$, $\hat{\phi}^{n+1} = z \hat{\phi}^n$ and dropping the index $n+1$ we obtained
\begin{equation}
(\partial_x^2 - \bar{\mu}^2) \hat{\textbf{u}} = \dfrac{z \Delta t}{z + 1} [- (\partial_x^2 - \lambda^2) + \dfrac{2 R \, Q(z)}{\Delta t}] \nabla \hat{\phi}
\end{equation}\\
where again we have defined $\bar{\mu}$ to be the positive real part of $\bar{\mu}^2 = k^2 + R \, \rho$ and $\rho = \dfrac{2(z - 1)}{\Delta t (z + 1)}$ and $\lambda^2 = k^2 + \dfrac{2 R}{\Delta t}$ \cite{brown2001accurate}.\\

With the solution of $\hat{\phi}$ obtained earlier \eqref{eq:solution of numerical phi} we can then solve for velocities accordingly
\begin{equation}\label{eq:solve for u using laplace of u}
(\partial_x^2 - \bar{\mu}^2) \hat{\textbf{u}} = \dfrac{z \Delta t}{z + 1} [- (\partial_x^2 - \lambda^2) + \dfrac{2 R \, Q(z)}{\Delta t}] \nabla \,(\hat{\phi}_h + \hat{\phi}_p)
\end{equation}
Expanding the right hand side and using the definition of $\gamma^2$ and $\lambda^2$:\\
\begin{equation*}
\dfrac{z \Delta t}{z + 1} [- (\partial_x^2 - k^2) \nabla \hat{\phi}_h + \dfrac{2R}{\Delta t}(1+Q(z)) \nabla \hat{\phi}_h - \partial_x^2\,\nabla \hat{\phi}_p + \gamma^2\,\nabla \hat{\phi}_p] 
\end{equation*}\\
It is worth to note that most of the terms above can actually be dropped out including $\hat{\phi}_p$ which contains the spurious mode. Too see this recall $(\partial_x^2 - k^2)\hat{\phi} = \hat{\eta}$ and hence the gradient of $\hat{\phi}$ satisfies the equation too by an interchange of operators. Therefore $(\partial_x^2 - k^2) \nabla \hat{\phi}_h =0$; now recall the particular solution of $\hat{\phi}$ is: $A_2 \,e^{-\gamma x}$ hence we have:\\
\begin{equation*}
- \partial_x^2\,\nabla \hat{\phi}_p + \gamma^2\,\nabla \hat{\phi}_p = 
\begin{cases}
-\partial_x^2\,\partial_x \,(A_2 \,e^{-\gamma x}) + \gamma^2\,\partial_x\,(A_2 \,e^{-\gamma x})
= \gamma^3\,A_2 \,e^{-\gamma x} - \gamma^3\,A_2 \,e^{-\gamma x} = 0\\
-\partial_x^2\,ij \,(A_2 \,e^{-\gamma x}) + \gamma^2\,ik\,(A_2 \,e^{-\gamma x})
= -\gamma^2 \,ik\,A_2 \,e^{-\gamma x} + \gamma^2\,ik\,A_2 \,e^{-\gamma x} = 0
\end{cases}
\end{equation*}

These results indicates that the spurious mode contained in $\hat{\phi}_p$ is now eliminated from the right hand side of equation \eqref{eq:solve for u using laplace of u}, leaving a simpler equation to solve for $\hat{\textbf{u}}$:
\begin{equation}
(\partial_x^2 - \bar{\mu}^2) \hat{\textbf{u}} = \dfrac{2\, z (1+ Q(z))}{z + 1} \nabla \hat{\phi}_h
\end{equation}

For $\hat{u}$ component:
\begin{equation*}
(\partial_x^2 - \bar{\mu}^2) \hat{u} = - \dfrac{2 \, z (1+ Q(z))}{z + 1} \, |k| A_1 e^{- |k| x}
\end{equation*}
Solving this inhomogeneous ordinary differential equation we obtain
\begin{dmath*}\label{eq:solution of numerical u}
\hat{u} = U e^{-\bar{\mu} x} + \dfrac{G(z)}{\rho} |k| A_1 e^{- |k| x}
\end{dmath*}
where for convenience we have defined $G(z) = \dfrac{2z\,F}{1+z}$

Similar process can be used to solve for $\hat{v}$.\\

In summary we have:
\begin{equation}\label{eq:solutions of numerical u, v and phi}
\begin{cases}
\hat{u}_{nu} = U e^{-\bar{\mu} x} + \dfrac{G(z)}{\rho} \, |k| A_1 e^{- |k| x} \\
\hat{v}_{nu} = V e^{-\bar{\mu} x} - \dfrac{G(z)}{\rho} \, i k A_1 e^{- |k| x} \\
\hat{\phi}_{nu} = A_1 e^{- |k| x} + A_2 e^{- \gamma x} \\
\end{cases}
\end{equation}
where ``nu" refers to numerical solutions.\\

Now we can apply the boundary conditions to solve for the undetermined coefficients. Recall at $x = 0$, the exact solutions should satisfy: 
\begin{equation}\label{eq:equations for boundary conditions for u* in normal mode analysis}
u(0,y,t_n) =\alpha, \, \, \, v(0,y,t_n) = \beta,
\end{equation}
\begin{equation*}
\partial_x u(0,y,t_n) + \partial_y v(0,y,t_n) = 0
\end{equation*}
where $t_n = (n+1)\Delta t$\\
As implied by the projection the numerical solutions should satisfy
\begin{dmath}\label{eq:boundary conditions for intermediate velocity field in normal mode analysis}
u^* \,|_{x=0} = u^{n+1}\,|_{x = 0} = \alpha \condition{   $\phi_x^{n+1} \,|_{x = 0} = 0,\,\,\, $}
\end{dmath}
\begin{dmath*}
v^* \,|_{x = 0} = (v^{n+1} - \Delta t \partial_y\phi^{n+1}) \,|_{x = 0}
\end{dmath*}

The tangential boundary condition needs more attention because we actually don't have access to $\phi^{n+1}$ when solving for $\textbf{u}^*$. Hence approximation is needed. Introducing function $\mathcal{B} (\phi)$ which approximates $\phi^{n+1}$ ($\mathcal{B}$ depends on the particular projection methods). In practice, this implies the following tangential boundary condition for intermediate velocity:
\begin{dmath}\label{eq:tangential boundary conditions for intermediate velocity field in normal mode analysis}
v^* \,|_{x = 0} = (\beta - \Delta t \partial_y\, \mathcal{B}(\phi)) \,|_{x = 0}
\end{dmath}

It is important to point out that now the numerical velocity $v^{n+1}$ does not equal to the exact $v(t_n)$ along the boundary any more (except we enforce $\phi^{n+1}=0$ along the boundary). It rather approximates $v(t_n)$ to an order of accuracy depends on function $\mathcal{B}$. This is a well known phenomenon in projection method \cite{strikwerda1999accuracy,brown2001accurate}. 3 choices of $\mathcal{B}$ is made here with $\mathcal{B}=0,\,\phi^n,\,2\phi^n-\phi^{n-1}$ which corresponds to no approximation, first order and second order approximations to $\phi^{n+1}$. We will see that choice of the approximations has a critical impact on the accuracy of the projection methods. Combining the above equations \eqref{eq:tangential boundary conditions for intermediate velocity field in normal mode analysis} and \eqref{eq:boundary conditions for intermediate velocity field in normal mode analysis} and performing Fourier transformations we obtained:\\

\begin{dmath}
\hat{u}=\hat{\alpha}
\end{dmath}
\begin{dmath}
(\hat{v}^{n+1}  + ik \Delta t (\hat{B} - 1) \hat{\phi} )|_{x = 0} = \hat{\beta}
\end{dmath}

where we have introduced $\hat{B}\hat{\phi} = \hat{\mathcal{B}}$\\

\begin{dmath}
\partial_x \hat{\phi} |_{x = 0} = 0
\end{dmath}

Now let's compute the coefficients first\\
From \eqref{eq:equations for boundary conditions for u* in normal mode analysis} and the expression of $\hat{u}$ and $\hat{v}$ we obtained the following relations:
\begin{dgroup}\label{eq:expressions for coefficients in numerical projection method in normal mode analysis}
\begin{dmath}
U = \hat{\alpha} - \dfrac{G(z) \, |k|}{\rho} A_1
\end{dmath}
\begin{dmath}
\bar{\mu} U = ik V
\end{dmath}
\begin{dmath}
-|k| A_1 - \gamma A_2 = 0 \, \, \, \Rightarrow A_2 = - \dfrac{|k|}{\gamma} A_1
\end{dmath}
\begin{dmath}
V - \dfrac{ik \, G(z)}{\rho} A_1 + ik \Delta t (\hat{B} - 1) (A_1 + A_2) = \hat{\beta}
\end{dmath}
\end{dgroup}

We can then solve for $A_1$ by manipulating the above equations and the expressions for $U$ and $V$. Also need to use the relation introduced before:$\Delta t = \dfrac{2F}{\gamma^2-k^2}$. After some tedious algebra we obtained:
\begin{equation}
A_1 = E^{-1} (\bar{\mu} \hat{\alpha} - ik \hat{\beta})
\end{equation}
where we have defined: $E =  \dfrac{R \, G(z) |k|}{\bar{\mu} + |k|}(1 + \dfrac{C \, F(\hat{B} - 1)}{G(z)})$ and $C = \dfrac{2 |k|(\bar{\mu} + |k|)}{\gamma (\gamma + |k|)}$ for convenience.\\

$E$ represents precisely the coupling between the choice of pressure approximation ($G(z)$ and $F$) and choice of boundary condition of projection (this will affect $\hat{B}$). The coupling between these functions must be carefully chosen to maintain second order accuracy for all variables across the domain. We will see in the subsequent discussions how this can be met.\\

Then substitute the expression for $A_1$ back into Equations \eqref{eq:expressions for coefficients in numerical projection method in normal mode analysis} and we recover other coefficients too:\\
In summary:
\begin{equation}\label{eq:solutions of coefficients in numerical projection method in normal mode analysis}
\begin{cases}
A_1 = \dfrac{(\bar{\mu} + |k|)\,(\bar{\mu} \hat{\alpha} - ik \hat{\beta})}{R \, G(z) |k|}(1 + \dfrac{C \, F(\hat{B} - 1)}{G(z)})^{-1}\\
U = \hat{\alpha} - \dfrac{(\bar{\mu} \hat{\alpha} - ik \hat{\beta}) \, (\bar{\mu} + |k|)}{R \, \rho} (1 + \dfrac{C\,F(\hat{B} - 1)}{G(z)})^{-1}\\
A_2 = - \dfrac{(\bar{\mu} \hat{\alpha} - ik \hat{\beta}) \, (\bar{\mu} + |k|)}{\gamma \, R \rho G(z)}(1 + \dfrac{C\,F(\hat{B} - 1)}{G(z)})^{-1}\\
V = \dfrac{\bar{\mu} \hat{\alpha}}{i k} - \dfrac{\bar{\mu}(\bar{\mu} \hat{\alpha} - ik \hat{\beta}) \, (\bar{\mu} + |k|)}{ik \, R \, \rho} (1 + \dfrac{C\,F(\hat{B} - 1)}{G(z)})^{-1}\\
\end{cases}
\end{equation}
%U = \hat{\alpha} - \dfrac{R(z) |k|}{\rho} E^{-1}\,(\bar{\mu} \hat{\alpha} - ik \hat{\beta})
%= \hat{\alpha} - \dfrac{(\bar{\mu} \hat{\alpha} - ik \hat{\beta}) \, (\bar{\mu} + |k|)}{Re \, \rho} (1 + %\dfrac{C\,F(\hat{B} - 1)}{R(z)})^{-1}

For our purpose of testing the accuracy, we want to compare the expression between the reference solutions \eqref{eq:Reference solution for transformed linearised Stokes equations} and our numerical solutions \eqref{eq:solutions of numerical u, v and phi}.\\

\paragraph*{Error analysis of velocity} 
For $\hat{u}$ velocity, let's examine the accuracy of the particular solution component because both the numerical and the reference components have the same decaying rate ($- |k|$). Thus we compare
\begin{equation*}
\text{Numerical: }\dfrac{G(z) |k|}{\rho} A_1 \text{   with   reference: }\dfrac{(\mu + |k|)}{R \, s} (\mu \hat{\alpha} - ik \hat{\beta})
\end{equation*}
Followed from the result for $A_1$ we could expand the expression of the numerical solution as:
\begin{equation*}
\dfrac{G(z) |k|}{\rho} A_1 = \dfrac{G(z) |k|}{\rho} \, E^{-1} (\bar{\mu} \hat{\alpha} - ik \hat{\beta})
\end{equation*}

Therefore it is obvious that we want 
\begin{equation}
\dfrac{G(z) |k|}{\rho} \, E^{-1} = \dfrac{(\mu + |k|)}{R \, s} + \mathcal{O}(\Delta t^2) \, \text{ and } \, (\bar{\mu} \hat{\alpha} - ik \hat{\beta}) = (\mu \hat{\alpha} - ik \hat{\beta})+\mathcal{O}(\Delta t^2)
\end{equation}
where we have used the ``Big $\mathcal{O}$" notation to express the error.\\

Let's do the second one first since it is easier! Ideally we want
\begin{equation*}
\mu \hat{\alpha} - ik \hat{\beta} = \bar{\mu} \hat{\alpha} - ik \hat{\beta} + \mathcal{O}(\Delta t^2)
\end{equation*}

Observe that the only term inhibits the accuracy is $\bar{\mu}$. \\

Recall $\bar{\mu}^2 = k^2 + R \, \rho$ and  $\mu^2 = k^2 + R \, s$
Hence it is $\rho$ which produces the error. A rough estimate indicate that we want $\rho$ converging to $s$ at least with second order accuracy. Hence let's prove this hypothesis first!\\

Recall the definition of $\rho$ is $\rho = \dfrac{2(z-1)}{\Delta t (z+1)}$. Because by definition:
\begin{equation*}
z = e^{s\Delta t} = \sum_{n = 0}^\infty \dfrac{(s \Delta t)^n}{n !} = 1 + \mathcal{O}(\Delta t)
\end{equation*}
hence $(z-1)^n = \mathcal{O}(\Delta t^n)$
\\
By conducting Taylor expansion of $f(z) = \dfrac{z-1}{z+1}$ at $z = 1$ to order $(z-1)^3$ we obtained:
\begin{dmath*}
f(z) = \sum_{n=0}^\infty \dfrac{f^{(n)}(1)\,(z-1)^n}{n!}
= f(1) + f'(1)\,(z-1) + \dfrac{f''(1)\,(z-1)^2}{2} + \mathcal{O}(\Delta t^3)
= \dfrac{e^{s\Delta t} - 1}{2} - \dfrac{(e^{s\Delta t} - 1)^2}{4}
\end{dmath*}

By expanding $e^{s\Delta t}$ based on its definition we found
\begin{dmath*}
f(z) = \dfrac{s\Delta t}{2} - \dfrac{(s\Delta t)^3}{4} + \mathcal{O}(\Delta t^3)
\end{dmath*}

Then it is easy to show that

\begin{dmath}\label{eq:error for rho}
\rho = \dfrac{2f(z)}{\Delta t}
= s - \dfrac{s^3\Delta t^2}{2} + \mathcal{O}(\Delta t^2)
= s + \mathcal{O}(\Delta t^2)
\end{dmath}

As for the error between $\bar{\mu}$ and $\mu$, it is straightforward to use a similar Taylor series argument and the result proven for $\rho$ to show 
\begin{equation}\label{eq:error for mu}
\bar{\mu} = \mu + \mathcal{O}(\Delta t^2)
\end{equation}

These are the two important relations that we use in the accuracy test.\\

Now for our purpose of comparing numerical and reference solutions for $\hat{u}$, there are three varying functions $C$, $G(z)$ and $B(z)$ that we must consider to obtain second order accuracy.
\begin{equation*}
G(z) |k|E^{-1} = \dfrac{\bar{\mu} + |k|}{R}(1 + \dfrac{C \, F(\hat{B} - 1)}{G(z)})^{-1} \text{ compare with } \dfrac{\mu + |k|}{R} 
\end{equation*}

Hence according to the above formulation, we want $(1 + \dfrac{C \, F(\hat{B} - 1)}{G(z)})$ to converge to 1 (equivalently $\dfrac{C \, F(\hat{B} - 1)}{G(z)}$ to 0) at second order rate. By varying the pressure approximation function we can make this possible. Recall the 3 choices of $\mathcal{B}$ for projection methods
\begin{equation}
\begin{array}{lccl}
\mathcal{B} = 0 & \Rightarrow \hat{B}=0 & \text{   no approximation to $\phi^{n+1}$   Alg 1 and Alg 2}\\

\mathcal{B} = \phi^n & \Rightarrow \hat{B} = \dfrac{1}{z} & \text{   first order approximation to $\phi^{n+1}$  Alg 3}\\

\mathcal{B} = 2\phi^n - \phi^{n-1} & \Rightarrow \hat{B} = \dfrac{2}{z} - \dfrac{1}{z^2} &\text{second order approximation  Alg 3}\\
\end{array}
\end{equation}

Let's consider these methods separately here:\\

For Alg 1 and Alg 2 with $\mathcal{B} = 0$, we have:
\begin{equation*}
\vert \dfrac{C\,F(\hat{B} -1)}{G(z)}\vert = \vert \dfrac{C\,F}{G(z)}\vert
\end{equation*}
Substitute the expression for $C$ and using the identity: $| \gamma (\gamma + |k|) | \geqslant |\gamma^2|$ we obtained:
\begin{equation*}
|\dfrac{C\,F}{G(z)}| \leqslant |\,\dfrac{2|k|(\bar{\mu}+|k|)F}{\gamma^2G(z)}\,|
\end{equation*}
Recall the definition of $\gamma^2$ the above inequality becomes:
\begin{equation*}
\Biggl|\,\dfrac{2|k|(\bar{\mu}+|k|)F}{\left(k^2+ \dfrac{2F}{\Delta t}\right) G(z)}\,\Biggr| \leqslant \Biggl|\,\dfrac{|k|(\bar{\mu}+|k|)}{\left(\dfrac{G(z)}{\Delta t}\right)}\,\Biggr| = \Biggl|\,\dfrac{\Delta t \,|k|(\bar{\mu}+|k|)}{G(z)}\,\Biggr| 
\end{equation*}
For Alg 1 and 2 we know that $Q(z) = \dfrac{1}{z-1}$ and so
\begin{equation*}
G(z) = \dfrac{2z(1+Q(z)}{z+1} = \dfrac{2z^2}{z^2-1}
\end{equation*}
Then it is quite straightforward to show that 
\begin{equation*}
\dfrac{1}{G(z)} = \dfrac{z^2-1}{2z^2} = \dfrac{1}{2} - \dfrac{1}{2}z^{-2} = \mathcal{O}(\Delta t)
\end{equation*}
Then we finally have that:
\begin{equation}
\Biggl|\dfrac{C\,F}{G(z)}\Biggr| \leqslant \Biggl|\,\dfrac{\Delta t \,|k|(\bar{\mu}+|k|)}{G(z)}\,\Biggr| = \mathcal{O}(\Delta t^2)
\end{equation}

In addition since we have just proved $\bar{\mu} = \mu + \mathcal{O} (\Delta t^2)$, hence $G(z) |k|E^{-1}$ is a second order approximation to $\dfrac{\mu + |k|}{R} $. Therefore the boundary condition: $\textbf{u}^* \,|_{x=0} = \textbf{u}^{n+1}\,|_{x=0} = (\alpha,\beta)$ does give second order accuracy for the coefficients of the numerical solutions.\\

For Alg 3, the problem is more complicated because now the extra term $\dfrac{C \, F(\hat{B} - 1)}{G(z)}$ in the coefficient $A_1,\,A_2$ only converge to 0 at first order rate. \\

Similar to the analysis done for Alg 1 and 2 we know that:
\begin{equation*}
\left|\dfrac{C\,F}{G(z)}\right| \leqslant \left|\,\dfrac{\Delta t \,|k|(\bar{\mu}+|k|)}{G(z)}\,\right|
\end{equation*}
Because now $Q(z) = 0$ (since $q=0$ for Alg 3) and so $G(z) = \dfrac{2z}{z+1}$. This implies 
\begin{equation*}
\dfrac{1}{G(z)} = \dfrac{1+z}{2z} = \dfrac{1}{2} + \dfrac{1}{2}z = 1 + \mathcal{O}(\Delta t)
\end{equation*} 
Hence we then found that
\begin{equation}
\Biggl|\dfrac{C\,F}{G(z)}\Biggr| \leqslant \Biggl|\,\dfrac{\Delta t \,|k|(\bar{\mu}+|k|)}{G(z)}\,\Biggr| = \mathcal{O}(\Delta t)
\end{equation}
which is only first order!\\
This makes $G(z) |k|E^{-1} = \dfrac{\bar{\mu} + |k|}{R}(1 + \dfrac{C \, F(\hat{B} - 1)}{G(z)})^{-1} = \dfrac{\mu + |k|}{R} +\mathcal{O}(\Delta t)$ which will also make the $U$ velocity only first order accurate.\\

This therefore implies that we need more accurate approximation to $\phi^{n+1}$ along the boundary. Now consider a first order approximation corresponding to $\hat{B}=\dfrac{1}{z}$.\\

Then the extra term satisfies:
\begin{equation*}
\Biggl|\dfrac{C\,F\,\hat{B}-1}{G(z)}\Biggr| \leqslant \Biggl|\dfrac{C\,F}{G(z)}|\,|\hat{B}-1\Biggr|
\end{equation*}
Now we want $\hat{B}-1$ to be first order accurate at least. This is indeed true since:
\begin{equation}
\hat{B} - 1 = z^{-1} - 1 = \mathcal{O}(\Delta t)
\end{equation}

Similarly for $\mathcal{B} = 2 \phi^n - \phi^{n-1}$ which is a second order approximation to $\phi^{n+1}$ would result in $\hat{B} - 1 = \mathcal{O} (\Delta t^2)$, however first order accuracy is enough here.\\

Combine this we obtained the desired result:
\begin{equation}
\Biggl|\dfrac{C\,F\,(\hat{B}-1)}{G(z)}\Biggr| = \mathcal{O}(\Delta t^2)
\end{equation}

Hence for Alg 1, 2 and 3 we find that the following results hold:
\begin{equation}
\dfrac{G(z) |k|}{\rho} A_1 = \dfrac{(\mu + |k|)}{R \, s} (\mu \hat{\alpha} - ik \hat{\beta}) + \mathcal{O} (\Delta t^2)
\end{equation}\\

Now we can finally examine the accuracy for the first component of $\hat{u}$: compare 
$U e^{-\bar{\mu} x}$ numerical and $\dfrac{(\mu + |k|)}{R \,s} (- |k| \hat{\alpha} + ik \hat{\beta}) e^{-\mu x}$ reference solution.\\

First note since $\bar{\mu} = \mu + \mathcal{O} (\Delta t^2)$ then
\begin{equation*}
| e^{\bar{\mu}x} - e^{\mu x} | = | \sum_{n=0}^{\infty} \dfrac{(\bar{\mu} x)^n - (\mu x)^n}{n!} |
\end{equation*}
\begin{equation*}
= | 0 + (\bar{\mu}x - \mu x) + \dfrac{((\bar{\mu} x)^2 - (\mu x)^2)}{2} + \cdots
\end{equation*}
\begin{equation*}
\leq |\bar{\mu} - \mu| |x| + |\dfrac{((\bar{\mu} x)^2 - (\mu x)^2)}{2}| + \cdots
\end{equation*}
Neglecting higher error terms we obtained
\begin{equation*}
| e^{\bar{\mu}x} - e^{\mu x} | \leq \mathcal{O} (\Delta t^2)
\end{equation*}

As for their coefficients\\
\begin{equation*}
U = \hat{\alpha} - \dfrac{(\bar{\mu} \hat{\alpha} - ik \hat{\beta}) \, (\bar{\mu} + |k|)}{Re \, \rho} (1 + \dfrac{C\,F(\hat{B} - 1)}{G(z)})^{-1}
=\\
\dfrac{(\bar{\mu} + |k|) (-|k| \hat{\alpha} + ik \hat{\beta})}{Re \rho} (1 + \dfrac{C\,F(\hat{B} - 1)}{G(z)})^{-1}
\end{equation*}

Since we have shown that $(1 + \dfrac{C\,F(\hat{B} - 1)}{G(z)})^{-1} = 1 + \mathcal{O}(\Delta t^2)$ holds for all projection methods as long as appropriate boundary conditions are chosen, then we found that

\begin{equation}
U = \dfrac{(\bar{\mu} + |k|)\,(- |k|\hat{\alpha}+ik\hat{\beta}))}{R\,\rho} = \\
\dfrac{(\mu+|k|)\,(- |k|\hat{\alpha}+ik\hat{\beta}))}{Rs}+\mathcal{O}(\Delta t^2)
\end{equation}

Further, by combining the above equation with the expression of numerical $\hat{u}$ in \eqref{eq:solutions of numerical u, v and phi} and comparing with the reference solution in \eqref{eq:Reference solution for transformed linearised Stokes equations u component} we arrived at the following error result:
\begin{eqnarray*}
\hat{u}_{nu} &=& U e^{-\bar{\mu}x} + \dfrac{G(z)}{\rho}|k|A_1\,e^{-|k|x} \\
&=& \dfrac{(\mu+|k|)\,(-|k|\hat{\alpha}+ik\hat{\beta})}{Rs}\,e^{-\mu x} + \dfrac{(\mu+|k|)\,(\mu \hat{\alpha} - ik\hat{\beta})}{R\,s}\,e^{-|k|x} \\
&=&
\hat{u}_{ex} + \mathcal{O} (\Delta t^2)
\end{eqnarray*}

\paragraph*{The error analysis for pressure} is more complicated since the accuracy vary on different projection methods. Let's consider the 3 projection methods separately.\\

First for Alg 1\\
This corresponds to $q = p^{n-1/2}$, $L = I$ and $\mathcal{B} = 0 \, (\hat{B} = 1)$. The boundary condition for $\textbf{u}^{n+1}$ is satisfied exactly leading to $\textbf{u}^*\,|_{x=0} = \textbf{u}^{n+1}\,|_{x=0} = (\alpha((n+1)\Delta t),\,\beta((n+1)\Delta t))$.

Recall the transformed pressure update formula derived in \eqref{eq:transformed pressure update formula for Alg 1} we have:
\begin{equation}
\hat{p} = z^{1/2}\,\hat{p}^{n+1/2} = \dfrac{z^{3/2}}{z-1}\, A_1 e^{-|k|x} + \dfrac{z^{3/2}}{z-1}\, A_2 e^{-\gamma x})
\end{equation}
It is obvious that this formula would result in a degraded accuracy because the second term $\dfrac{z^{3/2}}{z-1}\, A_2 e^{-\gamma x}$ contains the spurious mode. It has the coefficient written explicitly as:
\begin{equation*}
\dfrac{z^{3/2}}{z-1}\, A_2  = -\dfrac{z^{3/2}}{z-1}\,\dfrac{(\bar{\mu} \hat{\alpha} - ik \hat{\beta})\,(\bar{\mu} + |k|)}{\gamma R \rho G(z)}
\end{equation*}

Clearly this term is not zero. In fact by using a Taylor series expansion at $z=1$ we found that $\dfrac{z^{3/2}}{z-1}\,A_2 = \mathcal{O} (\Delta t)$ and hence pressure will actually be first order accurate.\\

Second for Alg 2. The pressure approximation and boundary condition for $\textbf{u}^*$ are the same with Alg 1 however in this case we are using the more accurate pressure update formula. 

\begin{dmath}
\hat{p} = z^{1/2}\,(Q(z) + 1)\,A_1\,e^{-|k|x}
\end{dmath}

Also being $Q(z) = \dfrac{1}{z}$ the same as Alg 1, we have:
\begin{equation*}
G(z) = \dfrac{2z(1+Q(z))}{1+z} = \dfrac{2z^2}{z^2 - 1}
\end{equation*}

Combine with the expression of $A_1$ in \eqref{eq:solutions of coefficients in numerical projection method in normal mode analysis} we obtained:\\
\begin{equation}
\hat{p} = \dfrac{z+1}{2z^{1/2}}\,\dfrac{(\bar{\mu} + |k|)\,(\bar{\mu} \hat{\alpha}  -ik \hat{\beta})}{R\,|k|}\,(1 - \dfrac{C\,F}{G(z)})^{-1}\,e^{-|k|x}
\end{equation}

and then recall the reference pressure solution given in \eqref{eq:Reference solution for transformed linearised Stokes equations p component}
\begin{equation*}
|\,\hat{p}_{nu} - \hat{p}_{ex}\,| =|\, \dfrac{z+1}{2 z^{1/2}} \dfrac{(\bar{\mu} + |k|)\,(\bar{\mu} \hat{\alpha}  -ik \hat{\beta})}{R\,|k|} \,(1 - \dfrac{C\,F}{G(z)})^{-1} - \dfrac{(\mu + |k|)\,(\mu \hat{\alpha}  -ik \hat{\beta})}{R\,|k|}\,|
\end{equation*}
By using a Taylor series expansion we found that $\dfrac{z+1}{2z^{1/2}} = 1 + \mathcal{O}(\Delta t^2)$ and combine with the result that $\bar{\mu} = \mu + \mathcal{O}(\Delta t^2)$, it is straightforward to show that $\hat{p}_{nu} = \hat{p}_{ex} + \mathcal{O}(\Delta t^2)$

Hence here we have proved that by eliminating the spurious mode, the new pressure update formula used in Alg 2 indeed provides second order accuracy in pressure.\\

For Alg 3, the same modified pressure update formula is used (as in Alg 2), so the spurious mode is again eliminated. The pressure approximation is now $q = 0$ leads to $Q(z) = 0$. The transformed version of pressure update formula therefore is:
\begin{equation*}
\hat{p} = z^{1/2}\,A_1\,e^{-|k|x}
\end{equation*}

We want the coefficient $A_1$ be at least second order accurate compared to the coefficient for the reference pressure solution. If $\hat{B}=1$ as in Alg 1 and 2, this leads to $\left(1+\dfrac{C\,F(\hat{B}-1)}{G(z)}\right) = 1+\mathcal{O}(\Delta t)$ and eventually lead to a first order accuracy in numerical pressure.\\

Thus considering more accurate boundary conditions, for instance: $\hat{B}=\dfrac{1}{z}$ then $\left(1+\dfrac{C\,F(\hat{B}-1)}{G(z)}\right)= 1+\mathcal{O}(\Delta t^2)$. Then the pressure is now written as
\begin{equation*}
\hat{p} = \dfrac{z+1}{2z^{1/2}}\,\dfrac{(\bar{\mu} + |k|)\,(\bar{\mu} \hat{\alpha}  -ik \hat{\beta})}{R\,|k|}\,\left(1+\dfrac{C\,F(\hat{B}-1)}{G(z)}\right)^{-1}\,e^{-|k|x}
\end{equation*}
which clearly gives second order accuracy.\\

Same result holds if second order accurate approximation to $\phi^{n+1}$ is used which corresponds to $\hat{B} = \dfrac{2}{z}- \dfrac{1}{z^2}$.\\

Hence our analysis show that Alg 3 (the pressure free projection method) needs at least first order approximation of $\phi^{n+1}$ when computing $\textbf{u}^*$ along the boundary. This is in agreement with the findings in literature \cite{brown2001accurate,strikwerda1999accuracy}. In addition, this is also supported by our numerical results.

\subsection{Normal mode analysis of Gauge method}
In this subsection, the accuracy of the Gauge method is explored. We will demonstrate that the introduction of consistent update of Gauge variable ($\textbf{m}$) leads to significant improvement over the accuracy in pressure simulations.\\

For simplicity let's take the periodic channel analysed for Projection methods before. The geometry is the same and hence we take the same boundary conditions for analytical velocities ($\textbf{u}$).\\
Recall the semi-discretised Gauge formulation for the linearised Stokes equations:
\begin{dgroup}\label{eq:semi-discretised linearised Stokes equations using Gauge method}
\begin{equation}
\dfrac{\textbf{m}^{n+1} -\textbf{m}^n}{\Delta t} = \dfrac{1}{2} \nabla^2\left(\textbf{m}^{n+1} + \textbf{m}^n\right)
\end{equation}
\intertext{By the periodic geometry, the boundaries at $y = 0,\,2\pi$ are periodic. Only the Dirichlet boundary at $x=0$ needs to be specified. At this boundary, the $x$ direction is normal and the $y$ direction is tangential. Hence we have\\
}
\begin{dmath}
u(0,y,t) = \alpha(y,t) \condition{ $v(0,y,t) = \beta(y,t)$}
\end{dmath}
\begin{dmath}
m_1(0,y,t) = \alpha(y,t) \condition{ $m_2(0,y,t) = v(0,y,t) + \partial_y \chi (0,y,t)$}
\end{dmath}
\intertext{\\
Performing projection\\
}
\begin{dmath}
\nabla^2 \chi^{n+1} = \nabla \cdot \textbf{m}^{n+1} \condition{   with $\partial_x \chi^{n+1} = 0$}
\end{dmath}
\begin{dmath}\label{eq:compatibility condition of Gauge method in normal mode analysis}
\textbf{u}^{n+1} = \textbf{m}^{n+1} - \nabla \chi^{n+1}
\end{dmath}
\intertext{\\
By making the normal boundary of the Gauge variable the same as the velocity $\textbf{u}^{n+1}$, we obtained a zero Neumann boundary condition for the auxiliary field $\chi^{n+1}$. Similar to the projection method, this is one of the most common boundary conditions for auxiliary fields.}
\end{dgroup}

Again the Reynolds number is made to be 1 for simplicity. Because the Gauge variable $\textbf{m}$ is being updated consistently, hence like velocities and pressure we have a nice relationship between the Gauge variables calculated at different time iterations: $\textbf{m}^{n+1} = z \textbf{m}^n$ (where $z$ again is the discrete Laplace transform variable). This was not possible for the intermediate velocity field shown in the projection methods because it is rather discarded after each iteration. We cannot really locate the variable in time and thus makes the tracking of its errors difficult.\\

In this case, we can solve the Gauge variables directly by performing Laplace and Fourier transforms.\\
\begin{equation*}
\dfrac{\hat{\textbf{m}} - \dfrac{\hat{\textbf{m}}}{z}}{\Delta t} = \dfrac{1}{2}\left(\partial_x^2 - k^2 \right)
\left(\hat{\textbf{m}} + \dfrac{\hat{\textbf{m}}}{z} \right)
\end{equation*}
After rearranging and define $\bar{\mu} = k^2 + \rho R$ where $\rho = \dfrac{2(z-1)}{\Delta t \,(z+1)}$, the transformed momentum equations (in components) are:

\begin{dgroup}
\begin{dmath}
\left( \partial^2_x - \bar{\mu}^2 \right) \, \hat{m}_1 = 0
\end{dmath}
\begin{dmath}
\left( \partial^2_x - \bar{\mu}^2 \right) \, \hat{m}_2 = 0
\end{dmath}
\end{dgroup}
Solving these 2 ordinary differential equations we obtain an expression for $\hat{textbf{m}}$:
\begin{dgroup}
\begin{dmath}
\hat{m}_1 = A_1\,e^{-\bar{\mu} x}
\end{dmath}
\begin{dmath}
\hat{m}_2 = A_2\,e^{-\bar{\mu} x}
\end{dmath}
\end{dgroup}
where $A_1$ and $A_2$ are constants to be determined.\\
Note that unlike the intermediate velocities in Projection methods, the Gauge variable actually does not contain any spurious mode! This would indeed lead to an improved accuracy since there is no numerical boundary in $\textbf{m}$.\\
This in turn guarantees the auxiliary field is also free of spurious modes. It is solved by taking the divergence of the equation of compatibility condition \eqref{eq:compatibility condition of Gauge method in normal mode analysis}:
\begin{equation}
\left(\partial_x^2 - k^2 \right)\,\hat{\chi} = \partial_x \hat{m}_1 + ik\hat{m}_2
\end{equation}
which then leads to an expression of $\hat{\chi}$:
\begin{dmath}
%\hat{\chi} = \dfrac{1}{\rho} \left(P\,e^{-|k|\,x} - \bar{\mu} \hat{m}_1 + ik\hat{m}_2 \right)
%= \dfrac{1}{\rho} \left(P\,e^{-|k|\,x} - A\,\bar{\mu} e^{-\bar{\mu} x} + ik\,B\,e^{-\bar{\mu} x} \right)
\hat{\chi} = P e^{-|k|x} + \dfrac{(-\bar{\mu}A_1 + ikA_2)}{\rho}e^{-\bar{\mu}x}
\end{dmath}

The velocities and pressure can then be recovered using the compatibility condition and the Gauge update formula respectively.\\

\begin{dgroup}\label{eq:numerical solution of u and v Guage method in normal mode analysis}
\begin{dmath}
\hat{u} = \hat{m}_1 - \partial_x \hat{\chi} = |k|Pe^{-|k|x} + \dfrac{\left(-k^2A_1 + ikA_2\bar{\mu} \right)}{\rho}e^{-\bar{\mu}x}
\end{dmath}
\begin{dmath}
\hat{v} = \hat{m}_2 - ik \hat{\chi} = -ikPe^{-|k|x} + \dfrac{\left(A_2\bar{\mu}^2 + ikA_1\bar{\mu} \right)}{\rho}e^{-\bar{\mu}x}
\end{dmath}
\end{dgroup}
The pressure can be recovered using the transformed Pressure update formula:
\begin{equation*}
\dfrac{\hat{p}}{\sqrt{z}} = \dfrac{1}{\Delta t}\left(1 - \dfrac{1}{z} \right)\,\hat{\chi} - \dfrac{1}{2} \left(\partial_x^2 - k^2\right)\left(1 + \dfrac{1}{z}\right)\,\hat{\chi}
\end{equation*}
 which leads to 
\begin{equation}\label{eq:numerical solution of p Guage method in normal mode analysis}
\hat{p} = \dfrac{z+1}{2\,z^{1/2}}\left(\rho\,Pe^{-|k|x}\right)
\end{equation}\\
Unlike in projection method, the spurious mode must be eliminated with special pressure update formulas, the velocities and pressure in Gauge method naturally leads to the exclusion of spurious modes. Hence this demonstrates that no numerical boundary layers are formed in Gauge method. This is also confirmed by our numerical studies presented in the results chapter.\\

Equipped with the boundary conditions given, we can then solve for the undetermined coefficients in the solutions above. The normal boundary component satisfies:
\begin{equation}
\hat{m}_1 \,|_{x=0} = \hat{u}\, |_{x=0} = \hat{\alpha}
\end{equation}

The tangential boundary condition for $\hat{\textbf{m}}$ needs more attention, because by compatibility condition, solving $\textbf{m}^{n+1}$ requires the knowledge of $\chi^{n+1}$ which we don't have access yet. An approximation to $\hat{\chi}^{n+1}$ is therefore needed and similar to the projection methods, we used the same function: $\hat{B}$ to represent the different approximations.\\
%\begin{equation*}
%\hat{B} = 1,\,\dfrac{1}{z},\,\dfrac{2z-1}{z^2}\,\,\, \text{   for 
%\end{equation*}
The difference between the exact and actual conditions is shown below:
\begin{dgroup}
\begin{dmath}
\hat{u}\,|_{x=0} = \hat{m}_2\,|_{x=0} - ik \hat{\chi}^{n+1}\,|_{x=0} = \hat{\beta} \condition{ Exact relation}
\end{dmath}
\begin{dmath}
\hat{m}_2\,|_{x=0} = \hat{u}\,|_{x=0} + ik \hat{B}\hat{\chi}^{n+1}\,|_{x=0} \condition{ Actual boundary condition used in practice}
\end{dmath}
\end{dgroup}
By combining the above two equations and eliminate $\hat{m}_2$ we obtained:
\begin{equation}
\hat{u}\,|_{x=0} + ik \hat{\chi}\,(\hat{B}-1)\,|_{x=0} = \hat{\beta}
\end{equation}

Same problem in regarding to the choice of auxiliary field approximation function $\hat{B}$. Hence it is worth to note that the tangential velocity field now does not satisfy the right boundary condition exactly (but to an order of accuracy depending on the choice of $\phi^{n+1}$ approximation). Hence we now need to ensure it approaches $\beta$ along the boundary at a second order rate. Hence second order approximation to $\phi^{n+1}$ is needed and this corresponds to $\hat{B} = \dfrac{2z-1}{z^2}$ (For Alg 3 first order approximation to $\phi^{n+1}$ is sufficient since the approximation term is a multiplied with $\Delta t$ by the compatibility equation).\\

Hence along the boundary (taking $x=0$) we note the Gauge variables must satisfy:
\begin{dgroup}
\begin{dmath}
A_1 = \hat{\alpha} \condition{   $m_1$ at $x=0$}
\end{dmath}
\begin{dmath}
\hat{\beta} = A_2 - ik\left(\dfrac{2z-1}{z^2}\right)\left(P + \dfrac{(-\bar{\mu}A_1 + ikA_2)}{\rho} \right)
\end{dmath}
\end{dgroup}

Further because the auxiliary field satisfies a zero Neumann normal boundary condition ($\partial_x\hat{\chi} = 0$) hence we have:
\begin{equation}
\bar{\mu}^2A_1 - ik\bar{\mu}A_2 - |k|\rho\,P = 0
\end{equation}

Summaries these conditions we can solve for $A_1,\,A_2$ and $P$ directly in a linear system:
\begin{equation}
\begin{bmatrix}
\bar{\mu}^2/R & -ik\bar{\mu}/R & -|k|\rho \\
ik\bar{\mu}\hat{B}(z)/R & \left(\rho + k^2\hat{B}(z)/R\right) & -ik\hat{B}(z)\rho\\
1 & 0 & 0 \\
\end{bmatrix}
\begin{bmatrix}
A_1\\
A_2\\
P\\
\end{bmatrix}
= \begin{bmatrix}
0\\
\hat{\beta}\rho\\
\hat{\alpha}\\
\end{bmatrix}
\end{equation}
where $\hat{B}(z) = \dfrac{2z-1}{z^2}$. Further it can be shown by a Taylor expansion argument that $\dfrac{1}{\hat{B}(z)} = \dfrac{z^2}{2z-1} = 1 + \mathcal{O}(\Delta t^2)$. Substitute this back into the linear system above we can then solve for $A_1,\,A_2$ and $P$ easily. The solutions are:

\begin{dgroup}
\begin{dmath}
A_1 = \hat{\alpha}
\end{dmath}
\begin{dmath}
A_2 = \dfrac{\bar{\mu}^2\hat{\alpha} - (\bar{\mu} + |k|)(\bar{\mu}\hat{\alpha} - ik\hat{\beta})}{ik\bar{\mu}} + \mathcal{O}(\Delta t^2)
\end{dmath}
\begin{dmath}
P = \dfrac{(\bar{\mu} + |k|)(\bar{\mu}\hat{\alpha} - ik\hat{\beta})}{\rho \,|k|} + \mathcal{O}(\Delta t^2)
\end{dmath}
\end{dgroup}

The numerical solutions can then be recovered using Equations \eqref{eq:numerical solution of u and v Guage method in normal mode analysis} and \eqref{eq:numerical solution of p Guage method in normal mode analysis} by directly substituting $A_1,\,A_2$ and $P$ into the expressions for $\hat{u}, \,\hat{v}$ and $\hat{p}$. The solutions are finally summarised below:
\begin{equation}
\begin{cases}
\hat{u}_{nu} = \left(\dfrac{(\bar{\mu}+|k|)(-|k|\hat{\alpha}+ik\hat{\beta})}{R\rho} + \mathcal{O}(\Delta t^2) \right)e^{-\bar{\mu}x} + \left(\dfrac{(\bar{\mu} + |k|)(\bar{\mu}\hat{\alpha} - ik\hat{\beta})}{R\rho} + \mathcal{O}(\Delta t^2) \right)e^{-\bar{\mu}x}\\
\hat{v}_{nu} =  \left(\dfrac{i\bar{\mu}(\bar{\mu} + |k|)\,\left(- \dfrac{k}{|k|}\hat{\alpha}+i\hat{\beta}\right)}{R\rho} +\mathcal{O}(\Delta t^2)\right) e^{-\bar{\mu} x} +  \left(\dfrac{-ik(\bar{\mu} + |k|)(\bar{\mu}\hat{\alpha} - ik\hat{\beta})}{R\rho|k|} +\mathcal{O}(\Delta t^2)\right) e^{- |k| x} \\
\hat{p}_{nu} = \dfrac{z+1}{2z^{1/2}}\,\left(\dfrac{(\bar{\mu} + |k|)(\bar{\mu}\hat{\alpha} - ik\hat{\beta})}{R|k|} + \mathcal{O}(\Delta t^2) \right)e^{-\bar{\mu}x}\\
\end{cases}
\end{equation}
where subscript ``nu" denotes ``numerical" to distinguish from the analytical solutions in \eqref{eq:Reference solution for transformed linearised Stokes equations}.\\

Now recall from the subsection Projection methods, we have shown that $\rho = s + \mathcal{O} (\Delta t^2), \, \bar{\mu} = \mu + \mathcal{O} (\Delta t^2)$ and $\dfrac{z+1}{2z^{1/2}} = 1 + \mathcal{O} (\Delta t^2)$. Hence it is straightforward to show that the coefficients in the numerical solutions above are of second order accuracy to that of the analytical solutions presented in \eqref{eq:Reference solution for transformed linearised Stokes equations}.\\

Hence with the Gauge variable tangential boundary equals to $\textbf{$\tau$}\cdot\left(\textbf{u}^{n+1} + (2\chi^n - \chi^{n-1})\right)$, the Gauge method shows completely second order accuracy in both velocities and pressure. Also because neither the Gauge variable nor the auxiliary field contains spurious modes, hence there is no special formula needed to filter out the numerical boundary layers as in Projection methods. This means the error convergence does not depend on the domain and boundary conditions as heavily as the projection methods which has fully second order accuracy only in special domains (e.g. periodic domains). The Gauge method exhibits the same order of accuracy in general domains too (see Shen' paper \cite{pyo2005normal,guermond2006overview} for details of proof).
