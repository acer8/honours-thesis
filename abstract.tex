\chapter*{Abstract}\label{abstract}

\addcontentsline{toc}{chapter}{Abstract}



This thesis is an investigation of the numerical modelling to the $2D$ Navier Stokes Equations of Incompressible flow using the well-known ``Projection method". Solving the Navier Stokes equations is never an easy task, a major difficulty lies in the coupling between velocity variables and pressure. The efficiency of solving this coupled equation is not satisfactory especially at the time of last century when the computation power was very limited. Hence the invention of Projection method or (Splitting method) by Alexandre Joel Chorin and Roger Temam independently in the last 1960s was ground breaking in the field of Computational Fluid Dynamics. It basically decouples the velocities and pressure which gives a significant improvement in computation efficiency. In this thesis, we go through the fundamental idea of Projection method which relies on the Helmholtz Hodge Decomposition theorem, Chorin's original method and all the way to recent second order accurate extensions. In addition, a derivation of Navier Stokes equations is also given in chapter 2.\\

This thesis is mainly focused on the behaviour of temporal error in projection methods, hence for the sake of simplicity, we have implemented finite difference discretisation scheme. Normal mode analysis was performed in a periodic channel to analyse the order of accuracy of projection methods theoretically. Our analysis shows in periodic channel all proposed second order schemes should show at least second order accuracy in both velocities and pressure, this is also confirmed by our numerical results. In more general domains e.g. Dirichlet boundary conditions, the projection methods however show a degraded accuracy in pressure to about 1.5 order. Only the Gauge method (which is a recent variation of projection methods proposed by E and Liu \cite{weinan2003gauge,brown2001accurate,guermond2006overview}), shows full second order error convergence in both velocity and pressure. We have also seen the pressure free projection method proposed by Kim and Moin must use a high order approximation to the auxiliary field along the tangential component in order to recover second order accuracy in pressure. In addition, we have observed an inconsistency in pressure approximation choice with the projection method and a modification was proposed to restore optimal convergence. Interesting flow problems like the Driven cavity problem is also presented for qualitative illustration of the performance of projection methods in case of unsteady, turbulent flows.
